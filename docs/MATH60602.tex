% Options for packages loaded elsewhere
\PassOptionsToPackage{unicode}{hyperref}
\PassOptionsToPackage{hyphens}{url}
%
\documentclass[
]{book}
\usepackage{lmodern}
\usepackage{amssymb,amsmath}
\usepackage{ifxetex,ifluatex}
\ifnum 0\ifxetex 1\fi\ifluatex 1\fi=0 % if pdftex
  \usepackage[T1]{fontenc}
  \usepackage[utf8]{inputenc}
  \usepackage{textcomp} % provide euro and other symbols
\else % if luatex or xetex
  \usepackage{unicode-math}
  \defaultfontfeatures{Scale=MatchLowercase}
  \defaultfontfeatures[\rmfamily]{Ligatures=TeX,Scale=1}
\fi
% Use upquote if available, for straight quotes in verbatim environments
\IfFileExists{upquote.sty}{\usepackage{upquote}}{}
\IfFileExists{microtype.sty}{% use microtype if available
  \usepackage[]{microtype}
  \UseMicrotypeSet[protrusion]{basicmath} % disable protrusion for tt fonts
}{}
\makeatletter
\@ifundefined{KOMAClassName}{% if non-KOMA class
  \IfFileExists{parskip.sty}{%
    \usepackage{parskip}
  }{% else
    \setlength{\parindent}{0pt}
    \setlength{\parskip}{6pt plus 2pt minus 1pt}}
}{% if KOMA class
  \KOMAoptions{parskip=half}}
\makeatother
\usepackage{xcolor}
\IfFileExists{xurl.sty}{\usepackage{xurl}}{} % add URL line breaks if available
\IfFileExists{bookmark.sty}{\usepackage{bookmark}}{\usepackage{hyperref}}
\hypersetup{
  pdftitle={Analyse multidimensionnelle appliquée},
  pdfauthor={Denis Larocque, Léo Belzile},
  hidelinks,
  pdfcreator={LaTeX via pandoc}}
\urlstyle{same} % disable monospaced font for URLs
\usepackage{color}
\usepackage{fancyvrb}
\newcommand{\VerbBar}{|}
\newcommand{\VERB}{\Verb[commandchars=\\\{\}]}
\DefineVerbatimEnvironment{Highlighting}{Verbatim}{commandchars=\\\{\}}
% Add ',fontsize=\small' for more characters per line
\usepackage{framed}
\definecolor{shadecolor}{RGB}{248,248,248}
\newenvironment{Shaded}{\begin{snugshade}}{\end{snugshade}}
\newcommand{\AlertTok}[1]{\textcolor[rgb]{0.94,0.16,0.16}{#1}}
\newcommand{\AnnotationTok}[1]{\textcolor[rgb]{0.56,0.35,0.01}{\textbf{\textit{#1}}}}
\newcommand{\AttributeTok}[1]{\textcolor[rgb]{0.77,0.63,0.00}{#1}}
\newcommand{\BaseNTok}[1]{\textcolor[rgb]{0.00,0.00,0.81}{#1}}
\newcommand{\BuiltInTok}[1]{#1}
\newcommand{\CharTok}[1]{\textcolor[rgb]{0.31,0.60,0.02}{#1}}
\newcommand{\CommentTok}[1]{\textcolor[rgb]{0.56,0.35,0.01}{\textit{#1}}}
\newcommand{\CommentVarTok}[1]{\textcolor[rgb]{0.56,0.35,0.01}{\textbf{\textit{#1}}}}
\newcommand{\ConstantTok}[1]{\textcolor[rgb]{0.00,0.00,0.00}{#1}}
\newcommand{\ControlFlowTok}[1]{\textcolor[rgb]{0.13,0.29,0.53}{\textbf{#1}}}
\newcommand{\DataTypeTok}[1]{\textcolor[rgb]{0.13,0.29,0.53}{#1}}
\newcommand{\DecValTok}[1]{\textcolor[rgb]{0.00,0.00,0.81}{#1}}
\newcommand{\DocumentationTok}[1]{\textcolor[rgb]{0.56,0.35,0.01}{\textbf{\textit{#1}}}}
\newcommand{\ErrorTok}[1]{\textcolor[rgb]{0.64,0.00,0.00}{\textbf{#1}}}
\newcommand{\ExtensionTok}[1]{#1}
\newcommand{\FloatTok}[1]{\textcolor[rgb]{0.00,0.00,0.81}{#1}}
\newcommand{\FunctionTok}[1]{\textcolor[rgb]{0.00,0.00,0.00}{#1}}
\newcommand{\ImportTok}[1]{#1}
\newcommand{\InformationTok}[1]{\textcolor[rgb]{0.56,0.35,0.01}{\textbf{\textit{#1}}}}
\newcommand{\KeywordTok}[1]{\textcolor[rgb]{0.13,0.29,0.53}{\textbf{#1}}}
\newcommand{\NormalTok}[1]{#1}
\newcommand{\OperatorTok}[1]{\textcolor[rgb]{0.81,0.36,0.00}{\textbf{#1}}}
\newcommand{\OtherTok}[1]{\textcolor[rgb]{0.56,0.35,0.01}{#1}}
\newcommand{\PreprocessorTok}[1]{\textcolor[rgb]{0.56,0.35,0.01}{\textit{#1}}}
\newcommand{\RegionMarkerTok}[1]{#1}
\newcommand{\SpecialCharTok}[1]{\textcolor[rgb]{0.00,0.00,0.00}{#1}}
\newcommand{\SpecialStringTok}[1]{\textcolor[rgb]{0.31,0.60,0.02}{#1}}
\newcommand{\StringTok}[1]{\textcolor[rgb]{0.31,0.60,0.02}{#1}}
\newcommand{\VariableTok}[1]{\textcolor[rgb]{0.00,0.00,0.00}{#1}}
\newcommand{\VerbatimStringTok}[1]{\textcolor[rgb]{0.31,0.60,0.02}{#1}}
\newcommand{\WarningTok}[1]{\textcolor[rgb]{0.56,0.35,0.01}{\textbf{\textit{#1}}}}
\usepackage{longtable,booktabs}
% Correct order of tables after \paragraph or \subparagraph
\usepackage{etoolbox}
\makeatletter
\patchcmd\longtable{\par}{\if@noskipsec\mbox{}\fi\par}{}{}
\makeatother
% Allow footnotes in longtable head/foot
\IfFileExists{footnotehyper.sty}{\usepackage{footnotehyper}}{\usepackage{footnote}}
\makesavenoteenv{longtable}
\usepackage{graphicx,grffile}
\makeatletter
\def\maxwidth{\ifdim\Gin@nat@width>\linewidth\linewidth\else\Gin@nat@width\fi}
\def\maxheight{\ifdim\Gin@nat@height>\textheight\textheight\else\Gin@nat@height\fi}
\makeatother
% Scale images if necessary, so that they will not overflow the page
% margins by default, and it is still possible to overwrite the defaults
% using explicit options in \includegraphics[width, height, ...]{}
\setkeys{Gin}{width=\maxwidth,height=\maxheight,keepaspectratio}
% Set default figure placement to htbp
\makeatletter
\def\fps@figure{htbp}
\makeatother
\setlength{\emergencystretch}{3em} % prevent overfull lines
\providecommand{\tightlist}{%
  \setlength{\itemsep}{0pt}\setlength{\parskip}{0pt}}
\setcounter{secnumdepth}{5}
% \usepackage[mathscr]{eucal}
\DeclareMathAlphabet{\mathcrl}{U}{rsfs}{m}{n}

% \DeclareMathAlphabet{\mathcal}{OMS}{cmsy}{m}{n}
\usepackage{booktabs}
\usepackage[left=1.2in, right= 1.2in]{geometry}
\makeatletter
\def\thm@space@setup{%
  \thm@preskip=8pt plus 2pt minus 4pt
  \thm@postskip=\thm@preskip
}
\makeatother

% \AtBeginDocument{

% \geometry{left=1.2in, right= 1.2in}
\usepackage{fourier}
\usepackage[french]{babel}
% }
\usepackage[]{natbib}
\bibliographystyle{apalike}

\title{Analyse multidimensionnelle appliquée}
\author{Denis Larocque, Léo Belzile}
\date{Version du 2020-01-31}

\usepackage{amsthm}
\newtheorem{theorem}{Théorème}[chapter]
\newtheorem{lemma}{Lemme}[chapter]
\newtheorem{corollary}{Corollaire}[chapter]
\newtheorem{proposition}{Proposition}[chapter]
\newtheorem{conjecture}{Conjecture}[chapter]
\theoremstyle{definition}
\newtheorem{definition}{Définition}[chapter]
\theoremstyle{definition}
\newtheorem{example}{Exemple}[chapter]
\theoremstyle{definition}
\newtheorem{exercise}{Exercice}[chapter]
\theoremstyle{remark}
\newtheorem*{remark}{Remarque}
\newtheorem*{solution}{Solution}
\let\BeginKnitrBlock\begin \let\EndKnitrBlock\end
\begin{document}
\frontmatter
\maketitle

{
\setcounter{tocdepth}{1}
\tableofcontents
}
\mainmatter
\newcommand{\bs}[1]{\boldsymbol{#1}}
\newcommand{\eps}{\varepsilon}
\newcommand{\Rlang}{\textsf{R}}
\newcommand{\SAS}{\textsf{SAS}}
\newcommand{\Sp}{\mathscr{S}}
\newcommand{\E}[1]{{\mathsf E}\left(#1\right)}
\newcommand{\Va}[1]{{\mathsf{Var}}\left(#1\right)}
\newcommand{\Cor}[1]{{\mathsf{Cor}}\left(#1\right)}
\newcommand{\I}[1]{{\mathbf 1}_{#1}}

\hypertarget{introduction}{%
\chapter{Introduction}\label{introduction}}

\hypertarget{survol-du-cours}{%
\section{Survol du cours}\label{survol-du-cours}}

\hypertarget{analyse-factorielle-exploratoire}{%
\subsection{Analyse factorielle exploratoire}\label{analyse-factorielle-exploratoire}}

On dispose de \(p\) variables \(X_1, \ldots, X_p\). Peut-on expliquer les interrelations (la structure de corrélation) entre ces variables à l'aide d'un certain nombre (moins de \(p\)) de facteurs latents (non observés)?

L'analyse factorielle est souvent utilisée pour analyser des questionnaires (construction d'échelles) comme dans l'exemple suivant.

\BeginKnitrBlock{example}
\protect\hypertarget{exm:unnamed-chunk-1}{}{\label{exm:unnamed-chunk-1} }Pour les besoins d'une enquête, on a demandé à 200 consommateurs adultes de répondre aux questions suivantes par rapport à un certain type de magasin:

Sur une échelle de 1 à 5,

\begin{enumerate}
\def\labelenumi{\arabic{enumi}.}
\tightlist
\item
  pas important
\item
  peu important
\item
  moyennement important
\item
  assez important
\item
  très important
\end{enumerate}

Pour vous, à quel point est-ce important\ldots

\begin{enumerate}
\def\labelenumi{\arabic{enumi}.}
\tightlist
\item
  que le magasin offre de bons prix tous les jours?
\item
  que le magasin accepte les cartes de crédit majeures (Visa, Mastercard)?
\item
  que le magasin offre des produits de qualité?
\item
  que les vendeurs connaissent bien les produits?
\item
  qu'il y ait des ventes spéciales régulièrement?
\item
  que les marques connues soient disponibles?
\item
  que le magasin ait sa propre carte de crédit?
\item
  que le service soit rapide?
\item
  qu'il y ait une vaste sélection de produits?
\item
  que le magasin accepte le paiement par carte de débit?
\item
  que le personnel soit courtois?
\item
  que le magasin ait en stock les produits annoncés?
\end{enumerate}

Pouvons-nous identifier un nombre restreint de facteurs (concepts, dimensions) qui pourraient bien rendre compte de la structure de corrélation entre ces 12 variables?
\EndKnitrBlock{example}

Buts:

\begin{itemize}
\tightlist
\item
  Décrire et comprendre la structure de corrélation d'un ensemble de variables à l'aide d'un nombre restreint de concepts (appelés facteurs).
\item
  Réduire le nombre de variables en créant une nouvelle variable par facteur. Ces nouvelles variables pourront par la suite être utilisées dans d'autres analyses (régression linéaire multiple par exemple).
\end{itemize}

\hypertarget{suxe9lection-de-variables-et-de-moduxe8les}{%
\subsection{Sélection de variables et de modèles}\label{suxe9lection-de-variables-et-de-moduxe8les}}

Dans plusieurs situations, on doit développer un modèle de prévision. Par exemple, on pourrait devoir développer un modèle pour:

\begin{itemize}
\tightlist
\item
  Détecter les faillites des clients (ou des entreprises)
\item
  Cibler les clients qui seront intéressés par une offre promotionnelle
\item
  Détecter les fraudes (par carte de crédit ou dans les rapports de revenus)
\item
  Prévoir si un client va nous quitter.
\end{itemize}

Il y a en général plusieurs variables explicatives potentielles, et aussi plusieurs types de modèles possibles (régression linéaire, réseaux de neurones, arbres de régression ou de classification, etc.). Dans ce chapitre, nous verrons des principes généraux et des outils afin de sélectionner des modèles performants, ou bien un sous-ensemble de variables avec un bon pouvoir prévisionnel.

\hypertarget{ruxe9gression-logistique}{%
\subsection{Régression logistique}\label{ruxe9gression-logistique}}

On cherche à expliquer le comportement d'une variable binaire \(Y\) (\(0-1\)), à l'aide de \(p\) variables quelconques \(X_1, \ldots, X_p\).

\BeginKnitrBlock{example}
\protect\hypertarget{exm:unnamed-chunk-2}{}{\label{exm:unnamed-chunk-2} }Une banque offre aux gens la possibilité de faire une demande de carte de crédit en ligne en promettant une approbation (conditionnelle) en quelques minutes seulement. Le tout est basé sur un modèle automatique de classification qui décide d'accorder ou non la carte (\(Y=1\) ou \(Y=0\)) en fonction des réponses fournies par les clients potentiels à différentes questions comme: quel est votre revenu annuel brut (\(X_1\)), avez-vous d'autres cartes de crédit (\(X_2\)), êtes-vous locataire ou propriétaire (\(X_3\)), etc\ldots
\EndKnitrBlock{example}

Buts:

\begin{itemize}
\tightlist
\item
  Comprendre comment et dans quelle mesure les variables \(\boldsymbol{X}\) influencent la catégorie d'appartenance de \(Y\).
\item
  Développer un modèle pour faire de la classification, c'est-à-dire, prévoir la catégorie d'appartenance de \(Y\) pour un nouveau sujet à partir des variables \(\boldsymbol{X}\).
\end{itemize}

\hypertarget{analyse-de-regroupements}{%
\subsection{Analyse de regroupements}\label{analyse-de-regroupements}}

On cherche à créer des groupes (« \emph{clusters} ») d'individus homogènes en utilisant \(p\) variables \(X_1, \ldots, X_p\).

\BeginKnitrBlock{example}
\protect\hypertarget{exm:unnamed-chunk-3}{}{\label{exm:unnamed-chunk-3} }
Cette méthode est utilisée en marketing pour la \textbf{segmentation de marché}, qui consiste en

\begin{quote}
\ldots définir des sous-groupes réunissant des consommateurs qui partagent les mêmes préférences ou qui réagissent de façon semblable à des variables de marketing\footnote{d'Astous, A. (2000). \emph{Le projet de recherche en marketing}, 2e édition. Chenelière/McGraw-Hill. }
\end{quote}
\EndKnitrBlock{example}

But:

\begin{itemize}
\tightlist
\item
  Combiner des sujets en groupes (interprétables) de telle sorte que les individus d'un même groupe soient les plus semblables possible par rapport à certaines caractéristiques et que les groupes soient les plus différents possible.
\end{itemize}

\hypertarget{analyse-de-survie}{%
\subsection{Analyse de survie}\label{analyse-de-survie}}

On s'intéresse au temps avant qu'un événement survienne. Par exemple :

\begin{itemize}
\tightlist
\item
  Temps qu'un client demeure abonné à un service offert par notre compagnie.
\item
  Temps de survie d'un individu après avoir été diagnostiqué avec un certain type de cancer.
\item
  Temps qu'un employé demeure au service de la compagnie.
\item
  Temps qu'une franchise demeure en activité.
\item
  Temps avant la faillite d'une entreprise (ou d'un particulier).
\item
  Temps avant le prochain achat d'un client.
\end{itemize}

On observe chaque sujet jusqu'à ce que l'une des deux choses suivantes se produise: l'événement survient avant la fin de la période d'observation ou bien l'étude se termine et l'événement n'est toujours pas survenu. Dans le premier exemple, l'événement correspond au fait d'interrompre son abonnement. On dispose donc d'une variable temps \(T\) pour chaque individu qui est soit censurée, soit non censurée. Si l'individu a expérimenté l'événement avant la fin de la période d'observation, la valeur de est non censurée. Si l'événement n'est toujours pas survenu à la fin de la période d'observation, la valeur de \(T\) est censurée. Pour chaque individu, on dispose également d'un ensemble de variables explicatives \(X_1, \ldots, X_p\).

But:

\begin{itemize}
\tightlist
\item
  Étudier les effets des variables explicatives sur le temps de survie et obtenir des prévisions du temps de survie.
\end{itemize}

\hypertarget{donnuxe9es-manquantes}{%
\subsection{Données manquantes}\label{donnuxe9es-manquantes}}

Il arrive fréquemment d'avoir des valeurs manquantes dans notre échantillon.

Simplement ignorer les sujets avec des valeurs manquantes et faire l'analyse avec les autres sujets conduit généralement à des estimations biaisées et à de l'inférence invalide.

Dans ce chapitre, nous verrons une méthode très générale afin de traiter les données manquantes, l'imputation multiple. Nous verrons comment elle peut être utilisée dans un contexte d'inférence et dans un contexte de prévision.

\hypertarget{analyse-factorielle-exploratoire-1}{%
\chapter{Analyse factorielle exploratoire}\label{analyse-factorielle-exploratoire-1}}

\hypertarget{introduction-1}{%
\section{Introduction}\label{introduction-1}}

On dispose de \(p\) variables \(X_1, \ldots, X_p\).

\begin{itemize}
\tightlist
\item
  Y a-t-il des groupements de variables?
\item
  Est-ce que les variables faisant partie d'un groupement semblent mesurer certains aspects d'un facteur commun (non observé)?
\end{itemize}

Un tel groupement peut être détecté si plusieurs variables sont très corrélées entre elles. Est-ce que la structure de corrélation entre les \(p\) variables peut être expliquée à l'aide d'un nombre restreint de facteurs?

\emph{Exemple de facteurs}: Habileté quantitative, habileté sociale, importance accordée à la qualité du service, importance accordée à la loyauté, habileté de leader, etc\ldots

L'analyse factorielle est aussi une méthode de réduction du nombre de variables. En effet, une fois qu'on a identifié les facteurs, on peut remplacer les variables individuelles par un résumé pour chaque facteur (qui est souvent la moyenne des variables qui font partie du facteur).

Pour faire une analyse factorielle, la taille d'échantillon devrait être d'au moins 10 fois le nombre de variables.

\hypertarget{rappels-sur-le-coefficient-de-corruxe9lation-linuxe9aire}{%
\section{Rappels sur le coefficient de corrélation linéaire}\label{rappels-sur-le-coefficient-de-corruxe9lation-linuxe9aire}}

On veut examiner la relation entre deux variables \(X_j\) et \(X_k\) et on dispose de \(n\) couples d'observations, où \(x_{i, j}\) (respectivement \(x_{i, k}\)) est la valeur de la variable \(X_j\) (\(X_k\)) pour le \(i\)e individu.

Le coefficient de corrélation linéaire entre \(X_j\) et \(X_k\), que l'on note \(r_{j, k}\), cherche à mesurer la force de la relation linéaire entre deux variables, c'est-à-dire à quantifier à quel point les observations sont alignées autour d'une droite. Le coefficient de corrélation est
\begin{align*}
r_{j, k} = \frac{\sum_{i=1}^n (x_{i, j}-\overline{x}_j)(x_{i, k} -\overline{x}_{k})}{\left\{\sum_{i=1}^n (x_{i, j}-\overline{x}_j)^2 \sum_{i=1}^n(x_{i, k} -\overline{x}_{k})^2\right\}^{1/2}}
\end{align*}

Les propriétés les plus importantes du coefficient de corrélation linéaire \(r\) sont les suivantes:

\begin{enumerate}
\def\labelenumi{\arabic{enumi})}
\tightlist
\item
  \(-1 \leq r \leq 1\);
\item
  \(r=1\) (respectivement \(r=-1\)) si et seulement si les \(n\) observations sont exactement alignées sur une droite de pente positive (négative). C'est-à-dire, s'il existe deux constantes \(a\) et \(b>0\) (\(b<0\)) telles que \(y_i=a+b x_i\) pour tout \(i=1, \ldots, n\).
\end{enumerate}

Règle générale,

\begin{itemize}
\tightlist
\item
  Plus la corrélation est près de 1, plus les points auront tendance à être alignés autour d'une droite de pente positive. Par conséquent, plus la valeur de \(X\) augmente, plus celle de \(Y\) aura tendance à augmenter et vice-versa.
\item
  Plus la corrélation est près de \(-1\), plus les points auront tendance à être alignés autour d'une droite de pente négative. Par conséquent, plus la valeur de \(X\) augmente, plus celle de \(Y\) aura tendance à diminuer et vice-versa.
\item
  Lorsque la corrélation est presque nulle, les points n'auront pas tendance à être alignés autour d'une droite. Il est très important de noter que cela n'implique pas qu'il n'y a pas de relation entre les deux variables. Cela implique seulement qu'il n'y a pas de \textbf{relation linéaire} entre les deux variables.
\end{itemize}

\hypertarget{exemple-de-questionnaire}{%
\section{Exemple de questionnaire}\label{exemple-de-questionnaire}}

Le questionnaire suivant porte sur une étude dans un magasin. Pour les besoins d'une enquête, on a demandé à 200 consommateurs adultes de répondre aux questions suivantes par rapport à un certain type de magasin sur une échelle de 1 à 5, où

\begin{enumerate}
\def\labelenumi{\arabic{enumi}.}
\tightlist
\item
  pas important
\item
  peu important
\item
  moyennement important
\item
  assez important
\item
  très important
\end{enumerate}

Pour vous, à quel point est-ce important\ldots

\begin{enumerate}
\def\labelenumi{\arabic{enumi}.}
\tightlist
\item
  que le magasin offre de bons prix tous les jours?
\item
  que le magasin accepte les cartes de crédit majeures (Visa, Mastercard)?
\item
  que le magasin offre des produits de qualité?
\item
  que les vendeurs connaissent bien les produits?
\item
  qu'il y ait des ventes spéciales régulièrement?
\item
  que les marques connues soient disponibles?
\item
  que le magasin ait sa propre carte de crédit?
\item
  que le service soit rapide?
\item
  qu'il y ait une vaste sélection de produits?
\item
  que le magasin accepte le paiement par carte de débit?
\item
  que le personnel soit courtois?
\item
  que le magasin ait en stock les produits annoncés?
\end{enumerate}

Une analyse factorielle cherchera à identifier automatiquement des groupes de variables qui sont fortement corrélées entre elles.

Les commandes \textbf{SAS} (ainsi que plusieurs commentaires) pour faire les analyses se trouvent dans le fichier \texttt{MATH60602\_cours3.sas}. Les statistiques descriptives ainsi que la matrice des corrélations sont obtenues en exécutant les lignes suivantes:

\begin{verbatim}
proc corr data=multi.factor2;
var x1-x12;
run;
\end{verbatim}

\begin{center}\includegraphics[width=0.9\linewidth]{figures/01-facto-e1} \end{center}

\begin{center}\includegraphics[width=0.9\linewidth]{figures/01-facto-e2} \end{center}

\hypertarget{description-du-moduxe8le-danalyse-factorielle}{%
\section{Description du modèle d'analyse factorielle}\label{description-du-moduxe8le-danalyse-factorielle}}

On dispose d'observations sur \(p\) variables \(X_1, \ldots, X_p\). Le modèle d'analyse factorielle fait l'hypothèse que ces variables dépendent linéairement d'un plus petit nombre \(m\) de variables aléatoires, \(F_1, \ldots, F_m\), appelées facteurs communs et de \(p\) termes d'erreurs (ou facteurs spécifiques) \(\varepsilon_1, \ldots, \varepsilon_p\), de moyenne \({\mathsf E}\left(\varepsilon_i\right)=0\) et de variance \({\mathsf{Var}}\left(\varepsilon_i\right)=\psi_i\) pour \(i=1, \ldots, p\). Spécifiquement, le modèle est
\begin{align*}
X_1 &= \mu_1 + l_{11}F_1 + l_{12} F_2 + \cdots + l_{1m}F_m + \varepsilon_1\\
X_2 &= \mu_2 + l_{21}F_1 + l_{22} F_2 + \cdots + l_{2m}F_m + \varepsilon_2\\
&\vdots \\
X_p &= \mu_p + l_{p1}F_1 + l_{p2} F_2 + \cdots + l_{pm}F_m + \varepsilon_p, 
\end{align*}
où \(\mu_i\) est l'espérance de la variable aléatoire \(X_i\) (\(i=1, \ldots, p\)) et où \(l_{ij}\) est le chargement de la variable \(X_i\) sur le facteur \(F_j\) (\(i=1, \ldots, p\); \(j=1, \ldots, m\)).

Les espérances (\(\mu_i\)), les chargements (\(l_{ij}\)) et les variances (\(\psi_i\)) sont des quantités fixes, mais inconnues, tandis que les facteurs communs (\(F_i\)) et spécifiques (\(\varepsilon_i\)) sont des variables aléatoires non observables que l'on assume non corrélée aux facteurs \(F\) et entre elles.

Des hypothèses supplémentaires sont nécessaires afin de pouvoir utiliser ce modèle (contraintes d'identifiabilité des paramètres). Sans entrer dans les détails, mentionnons que l'une de ces hypothèses est que les facteurs sont non corrélés.

De plus, si les variables ont été préalablement standardisées de telle sorte que \({\mathsf E}\left(X_i\right)=0\) et \({\mathsf{Var}}\left(X_i\right)=1\) (note: ceci revient à utiliser la matrice de corrélation des observations dans l'analyse ce qui est fait par défaut dans \textbf{SAS}), alors \({\mathsf{Cor}}\left(X_i, F_j\right)=l_{ij}\), c'est-à-dire, le chargement de la variable \(X_i\) sur le chargement \(F_j\) est le coefficient de corrélation entre cette variable et ce facteur.

Sans aucune contrainte sur le modèle, la matrice de covariance de \(X_1, \ldots, X_p\) possède \(p(p+1)/2\) paramètres, soit \(p\) variances et \(p(p-1)/2\) termes de corrélation. Avec le modèle d'analyse factorielle, on suppose que l'on peut décrire cette structure en utilisant seulement \(p(m+1)\) paramètres (\(p\) variances spécifiques et \(pm\) chargements). Par exemple, avec \(p=50\) variables et \(m=6\) facteurs, on essaie de décrire la structure de covariance à l'aide de 350 paramètres au lieu de 1275.

Il existe plusieurs méthodes pour extraire les facteurs, c'est-à-dire pour estimer les paramètres du modèle (les \(\psi_i\) et les \(l_{ij}\)). Nous allons discuter de deux d'entre elles: la méthode du maximum de vraisemblance et la méthode des composantes principales. L'avantage de l'estimation par maximum de vraisemblance est qu'elle permet l'utilisation de critères d'information et de statistiques de tests pour guider le choix du nombre de facteurs. En revanche, l'estimation des paramètres requiert une optimisation numérique qui peut être délicate selon les cas de figure.

\hypertarget{rotation-des-facteurs}{%
\subsection{Rotation des facteurs}\label{rotation-des-facteurs}}

Dans le modèle d'analyse factorielle, on peut montrer que, lorsqu'il y a deux facteurs ou plus, il existe plusieurs configurations de facteurs qui donnent la même structure de covariance. En fait, les chargements peuvent seulement être déterminés à une transformation orthogonale prêt (note: une transformation orthogonale est une transformation qui préserve le produit scalaire; elle préserve ainsi toutes les distances et les angles entre deux vecteurs). Si les chargements provenant d'une méthode d'extraction des facteurs ne sont pas uniques, la matrice de corrélation estimée par le modèle est par contre unique.

Il existe plusieurs techniques de rotation de facteurs. Le but de ces techniques est d'essayer de trouver une solution qui fera en sorte que les facteurs seront facilement interprétables. La méthode la plus utilisée est la méthode \textbf{varimax}: elle produit une configuration de chargement en maximisant la variance de la somme des carrés des chargements pour les \(m\) facteurs. La méthode varimax tend à produire une configuration de facteurs tel que les chargements de chaque variable sont dispersés (des chargements élevés positifs ou négatifs et d'autres presque nuls).

Je vous suggère de toujours tenter d'interpréter la solution avec une rotation varimax. Si ce n'est pas suffisamment clair, il existe d'autres méthodes de rotation dont certaines (les rotations de type oblique) permettent la présence de corrélation entre les facteurs.

\hypertarget{estimation-des-facteurs}{%
\section{Estimation des facteurs}\label{estimation-des-facteurs}}

Les chargements estimés pour la solution à quatre facteurs, suite à la rotation varimax, sont obtenus avec le code SAS suivant:

\begin{verbatim}
proc factor data=multi.factor2 
 method=ml rotate=varimax nfact=4
 maxiter=500 flag=.3 hey;
 var x1-x12;
run;
\end{verbatim}

\begin{center}\includegraphics[width=0.7\linewidth]{figures/01-facto-e3} \end{center}

En général, on associe une variable à un groupe (facteur) si son chargement est supérieur à 0, 3 (en valeur absolue), ce qui donne

\begin{itemize}
\tightlist
\item
  Facteur 1: \(X_4\), \(X_8\) et \(X_{11}\)
\item
  Facteur 2: \(X_3\), \(X_6\), \(X_9\) et \(X_{12}\)
\item
  Facteur 3: \(X_2\), \(X_7\) et \(X_{10}\)
\item
  Facteur 4: \(X_1\) et \(X_5\).
\end{itemize}

Ces facteurs sont interprétables:

\begin{itemize}
\tightlist
\item
  Le facteur 1 représente l'importance accordée au service.
\item
  Le facteur 2 représente l'importance accordée aux produits.
\item
  Le facteur 3 représente l'importance accordée à la facilité de paiement.
\item
  Le facteur 4 représente l'importance accordée aux prix.
\end{itemize}

Dans cet exemple, les choses se sont bien passées et le nombre de facteurs que nous avons spécifié (4) semble être adéquat, mais ce n'est pas toujours aussi évident. Il est utile d'avoir des outils pour guider le choix du nombre de facteurs.

\hypertarget{choix-du-nombre-de-facteurs}{%
\section{Choix du nombre de facteurs}\label{choix-du-nombre-de-facteurs}}

Il existe différentes méthodes pour se guider dans le nombre de facteurs, \(m\), à utiliser. Cependant, le point important à retenir est que, peu importe le nombre choisi, il faut que les facteurs soient \textbf{interprétables}. Par conséquent, les méthodes qui
suivent ne devraient servir que de guide et non pas être suivies aveuglément.
La méthode du maximum de vraisemblance que nous avons utilisée dans l'exemple possède l'avantage de fournir trois critères pour choisir le nombre de facteurs appropriés. Ces critères sont:

\begin{itemize}
\tightlist
\item
  AIC (critère d'information d'Akaike)
\item
  BIC (critère d'information bayésien de Schwarz)
\item
  Le test du rapport de vraisemblance pour l'hypothèse nulle que le modèle de corrélation décrit le modèle factoriel avec \(m\) facteurs est adéquat, contre l'alternative qu'il n'est pas adéquat.
\end{itemize}

Les critères d'information servent à la sélection de modèles; ils seront traités plus en détail dans les chapitres qui suivent. Pour l'instant, il est suffisant de savoir que le modèle avec la valeur du critère AIC (ou BIC) la plus petite est considéré le « meilleur » (selon ce critère).

Les sorties suivantes proviennent du même programme SAS et correspondent au modèle factoriel avec quatre facteurs estimé par maximum de vraisemblance.

\begin{center}\includegraphics[width=0.7\linewidth]{figures/01-facto-e4} \end{center}

Pour choisir le nombre de facteurs avec les critères d'information, il faut ajuster le modèle en faisant varier le nombre de facteurs (option \texttt{nfact}) et extraire la valeur numérique correspondante.

Le tableau \ref{tab:ICtable} présente les valeurs estimées des critères d'information et des valeurs-\(p\) pour le test du rapport de vraisemblance pour cinq modèles. Le critère AIC suggère quatre facteurs, tandis que les deux autres critères (BIC et test du rapport de vraisemblance suggèrent plutôt trois facteurs.

\begin{longtable}[]{@{}crrr@{}}
\caption{\label{tab:ICtable} Critères d'information et valeurs-\emph{p} pour le modèle factoriel à \emph{m} facteurs}\tabularnewline
\toprule
m & AIC & BIC & valeur-\emph{p}\tabularnewline
\midrule
\endfirsthead
\toprule
m & AIC & BIC & valeur-\emph{p}\tabularnewline
\midrule
\endhead
1 & 228, 0 & 49, 9 & \textless0, 001\tabularnewline
2 & 99, 5 & -42, 3 & \textless0, 001\tabularnewline
3 & -20, 5 & -129, 3 & 0, 096\tabularnewline
4 & -34, 9 & -114, 1 & 0, 973\tabularnewline
5 & -24, 8 & -77, 6 & 0, 975\tabularnewline
\bottomrule
\end{longtable}

On peut considérer le modèle avec trois facteurs: les chargements (après rotation varimax) sont données dans la figure \ref{fig:fig1p5}.

\begin{figure}

{\centering \includegraphics[width=0.55\linewidth]{figures/01-facto-e5} 

}

\caption{Estimés des chargements pour trois facteurs avec rotation varimax}\label{fig:fig1p5}
\end{figure}

Cette solution récupère les trois facteurs \emph{service}, \emph{produits} et \emph{paiement} de la solution précédente à quatre facteurs. Le facteur \emph{prix} (qui était formé de \(X_1\) et \(X_5\)) n'est plus présent: que faire avec ce dernier? Cela dépend du but de l'analyse et nous y reviendrons plus tard.

Pour terminer cette section, voici la description de deux autres
critères \emph{classiques} pour choisir le nombre de facteurs. Ces deux critères sont:

\begin{itemize}
\tightlist
\item
  Critère de Kaiser, un critère basé sur les valeurs propres. Avec une analyse en composantes principales basée sur la matrice des corrélations, la valeur propre associée à un facteur représente la partie de la variance totale qui est expliquée par ce facteur. Chaque variable compte pour un dans la variance totale. Le nombre de facteurs choisis est le nombre de valeurs propres supérieures à 1. L'idée est de garder seulement les facteurs qui expliquent plus de variance qu'une variable individuelle.
\item
  le diagramme d'éboulis: un graphique des valeurs propres ordonnées de la plus grande à la plus petite en fonction de \(1, \ldots, p\). Habituellement, ce graphe prendra la forme d'une chute assez importante suivie d'une stabilisation des valeurs propres. Avec ce critère, le nombre de facteurs est déterminé par le nombre de valeurs propres avant le début du coude où il a stabilisation apparente. L'idée est de choisir l'endroit où l'ajout d'un facteur supplémentaire n'apporte qu'un gain marginal faible. Ce critère est par contre subjectif et dépend de l'analyste. En ajoutant \texttt{scree} comme option à \texttt{proc\ factor}, on obtient le diagramme d'éboulis mais il est facile de le créer manuellement et le résultat est esthétiquement plus réussi.
\end{itemize}

Les sorties qui suivent proviennent du programme:

\begin{verbatim}
proc factor data=multi.factor2 method=principal
 scree rotate=varimax flag=.3;
 ods output Eigenvalues=eigen;
 var x1-x12;
run; 

proc sgplot data=eigen;
 scatter x=number y=eigenvalue;
 yaxis label="valeurs propres";
 xaxis label='nombre';
run;
\end{verbatim}

Cette fois-ci, c'est la méthode des composantes principales qui est utilisée; cette dernière consiste à estimer les chargements en utilisant les \(m\) premières valeurs propres et vecteurs propres de la matrice de corrélation. En ne spécifiant pas l'option \texttt{nfact}, \textbf{SAS} choisit le nombre de facteurs en utilisant par défaut le critère de Kaiser (valeurs propres supérieures à 1). Quatre facteurs sont retenus, tel qu'indiqué par la sortie au bas du tableau \ref{fig:fig1p7}. Pour le diagramme d'éboulis de la figure \ref{fig:fig1p6}, le choix est assez subjectif: il semble raisonnable de choisir trois ou quatre facteurs.

\begin{figure}

{\centering \includegraphics[width=0.65\linewidth]{figures/01-facto-e7} 

}

\caption{Valeurs propres et proportion de variance}\label{fig:fig1p7}
\end{figure}

\begin{figure}

{\centering \includegraphics[width=0.65\linewidth]{figures/01-facto-e6} 

}

\caption{Diagramme d'éboulis}\label{fig:fig1p6}
\end{figure}

On suggère d'utiliser \emph{de facto} les trois critères découlant de l'utilisation de la vraisemblance et de déterminer le nombre de facteurs à extraire selon différents critères avant d'examiner les modèles avec ce nombre de facteurs et ceux
avec un facteur de moins ou de plus. Au final, le plus important est de pouvoir interpréter raisonnablement les facteurs et donc le modèle retenu est souvent choisi selon le critère \textbf{Wow!}. On veut dire par là que la configuration de facteurs choisie est compréhensible.

\hypertarget{construction-duxe9chelles-uxe0-partir-des-facteurs}{%
\section{Construction d'échelles à partir des facteurs}\label{construction-duxe9chelles-uxe0-partir-des-facteurs}}

Si le seul but de l'analyse factorielle est de comprendre la structure de corrélation entre les variables, alors se limiter à l'interprétation des facteurs est suffisant.

Si par contre, le but est de réduire le nombre de variables pour pouvoir par la suite procéder à d'autres analyses statistiques, l'analyse factorielle peut alors servir de guide pour construire de nouvelles variables (échelles). En supposant que l'analyse factorielle a produit des facteurs qui sont interprétables et satisfaisants, la méthode de construction d'échelles la plus couramment utilisée consiste à construire \(m\) nouvelles variables, une par facteur. Pour un facteur donné, la nouvelle variable est simplement la moyenne des variables ayant des chargements élevés sur ce facteur (positifs ou négatifs, mais de même signe). Une autre méthode, les scores factoriels, sera présentée plus loin.

Lorsqu'on construit une échelle, il est important d'examiner sa cohérence interne. Ceci peut être fait à l'aide du coefficient alpha de Cronbach. Ce coefficient mesure à quel point chaque variable faisant partie d'une échelle est corrélée avec le total de toutes les variables pour cette échelle.
Plus le coefficient est élevé, plus les variables ont tendance à être corrélées entre elles. L'alpha de Cronbach est
\begin{align*}
\alpha=\frac{k}{k-1} \frac{S^2-\sum_{i=1}^k S_i^k}{S^2}, 
\end{align*}
où \(k\) est le nombre de variables dans l'échelle, \(S^2\) est la variance empirique de la somme des variables et \(S_i^2\) est la variance empirique de la \(i\)e variable. En pratique, on voudra que ce coefficient soit au moins égal à 0, 6 pour être satisfait de la cohérence interne de l'échelle.

Avec \textbf{SAS}, la procédure \texttt{corr} permet de calculer \(\alpha\).

\begin{verbatim}
/* pour le facteur service */
proc corr data=multi.factor2 alpha;
var x4 x8 x11;
run;
/* pour le facteur produits */
proc corr data=multi.factor2 alpha;
var x3 x6 x9 x12;
run;
/* pour le facteur paiement */
proc corr data=multi.factor2 alpha;
var x2 x7 x10;
run;
/* pour le facteur prix */
proc corr data=multi.factor2 alpha;
var x1 x5;
run;
\end{verbatim}









\begin{figure}

{\centering \includegraphics[width=0.85\linewidth]{figures/01-facto-e8} 

}

\caption{Alpha de Cronbach pour le facteur \emph{service}.}\label{fig:fig1p8}
\end{figure}

\begin{figure}

{\centering \includegraphics[width=0.85\linewidth]{figures/01-facto-e9} 

}

\caption{Alpha de Cronbach pour le facteur \emph{produits}.}\label{fig:fig1p9}
\end{figure}

\begin{figure}

{\centering \includegraphics[width=0.85\linewidth]{figures/01-facto-e10} 

}

\caption{Alpha de Cronbach pour le facteur \emph{paiement}.}\label{fig:fig1p10}
\end{figure}

\begin{figure}

{\centering \includegraphics[width=0.85\linewidth]{figures/01-facto-e11} 

}

\caption{Alpha de Cronbach pour le facteur \emph{prix}.}\label{fig:fig1p11}
\end{figure}

Il faut utiliser le alpha brut. Ainsi, les alphas de Cronbach sont tous
satisfaisants (plus grand que 0, 6) sauf pour le facteur \emph{prix} (\(\alpha=0, 546\)). \textbf{SAS} fournit également la matrice des corrélations des variables de l'échelle ainsi que la valeur du alpha de Cronbach si on retirait une variable à la fois de l'échelle. Tout est donc cohérent. Les échelles provenant des facteurs \emph{service}, \emph{produits} et \emph{paiement}, sont satisfaisantes. Ces facteurs sont identifiés à la fois dans la solution à quatre, mais aussi dans la solution à troisfacteurs. Le facteur \emph{prix} est celui qui apparaît en plus dans la solution à quatre facteurs. Il a une interprétation claire, mais son faible alpha ferait en sorte qu'il serait discutable de travailler avec l'échelle \emph{prix} dans d'autres analyses (du moins avec selon l'usage habituel du alpha).

\hypertarget{compluxe9ments-dinformation}{%
\section{Compléments d'information}\label{compluxe9ments-dinformation}}

\hypertarget{variables-ordinales}{%
\subsection{Variables ordinales}\label{variables-ordinales}}

Théoriquement, une analyse factorielle ne devrait être faite qu'avec des
variables continues. Par contre, en pratique, on l'utilise souvent aussi avec des variables ordinales (comme pour l'exemple portant sur le questionnaire) et même avec des variables binaires (0-1).

Dans ce genre de situation, on peut aussi utiliser d'autres mesures d'associations au lieu du coefficient de corrélation linéaire. Par exemple, on peut utiliser la corrélation polychorique, qui est une mesure de corrélation entre deux variables ordinales. La corrélation tétrachorique correspond au cas spécial de deux variables binaires.

Ma suggestion est d'utiliser la corrélation linéaire ordinaire avec des variables ordinales (même binaires). Si les résultats ne sont pas satisfaisants, on peut alors essayer avec d'autres mesures d'associations.

On peut refaire l'analyse des données portant sur le magasin dans \textbf{SAS} en utilisant la corrélation polychorique calculées par la procédure \texttt{corr} et en passant la sortie à la procédure \texttt{factor}.

\begin{verbatim}
proc corr data=multi.factor2 polychoric out=poly_corr;
var x1-x12;
run;

proc factor data=poly_corr
 method=ml rotate=varimax nfact=4
 maxiter=500 flag=.3 hey;
 var x1-x12;
run;
\end{verbatim}

Les chargements sont donnés dans la figure \ref{fig:fig1p12}. Les facteurs obtenus sont les mêmes qu'en utilisant les corrélations linéaires.

\begin{figure}

{\centering \includegraphics[width=0.65\linewidth]{figures/01-facto-e12} 

}

\caption{Chargements estimés pour la corrélation polychorique}\label{fig:fig1p12}
\end{figure}

\hypertarget{autres-muxe9thodes-dextractions-de-facteurs}{%
\subsection{Autres méthodes d'extractions de facteurs}\label{autres-muxe9thodes-dextractions-de-facteurs}}

Il n'y a pas de formule explicites pour l'estimation des paramètres avec la méthode du maximum de vraisemblance et un algorithme d'optimisation est nécessaire pour l'option des paramètres. Dans certains cas, l'algorithme peut terminer sans solution ou retourner un cas limite (où la variance est négative ou nulle). C'est le cas dans notre exemple avec quatre facteurs (solution de Heywood), bien que ce ne soit pas indiqué. La sortie \textbf{SAS} contient des informations sur la convergence de l'estimé: idéalement, on obtient la mention \texttt{Critère\ de\ convergence\ respecté}; autrement, essayez de varier le nombre. Un autre signe que l'algorithme n'a pas convergé est la présence de degrés de libertés négatifs pour le test du rapport de vraisemblance.

La méthode par les composantes principales (mentionnée lors de la présentation des valeurs propres et du diagramme d'éboulis a une solution explicite et peut donc dépanner si on n'arrive pas à obtenir le maximum de vraisemblance.

D'autres méthodes sont aussi disponibles dans \textbf{SAS} (voir la rubrique d'aide du logiciel) mais les deux méthodes mentionnées devraient être suffisantes pour la grande majorité des applications.

\hypertarget{autres-muxe9thodes-de-rotation-des-facteurs}{%
\subsection{Autres méthodes de rotation des facteurs}\label{autres-muxe9thodes-de-rotation-des-facteurs}}

Jusqu'à présent, nous avons utilisé la méthode de rotation orthogonale varimax. Il existe de nombreuses autres méthodes de rotations orthogonales telles, orthomax, quartimax, parsimax et equimax (voir la rubrique d'aide de \textbf{SAS}). Rappelez-vous que le modèle d'analyse factorielle de base suppose que les facteurs sont non corrélés. Les rotations de type obliques quant à elles permettent d'introduire de la corrélation entre les facteurs. Quelquefois, une telle rotation facilitera davantage l'interprétation des facteurs qu'une rotation orthogonale. \textbf{SAS} permet l'utilisation de plusieurs méthodes de rotation obliques qui sont documentées dans la rubrique d'aide. Notez qu'il faut être prudent lorsqu'on utilise une méthode de rotation oblique car il y aura trois matrices de chargements après rotation (coefficients de régression normalisés, corrélations semi-partielles ou corrélations). On suggère l'utilisation de la première, soit la représentation avec \textbf{coefficients de régression normalisés}. Il s'agit des coefficients de régression si on voulait prédire les variables à l'aide des facteurs. Ils indiquent donc à quel point chaque facteur est associé à chaque variable. Dans le cas d'une rotation orthogonale, ces trois matrices sont les mêmes et il s'agit de trois interprétations valides des chargements.

Le programme suivant fait une analyse factorielle avec quatre facteurs, mais en utilisant une rotation varimax oblique (option \texttt{rotate=obvarimax}).

\begin{verbatim}
proc factor data=multi.factor2
maxiter=500 flag=.3 hey;
var x1-x12;
run;
\end{verbatim}

La matrice des corrélations entres facteurs est donnée dans la figure \ref{fig:fig1p13} et les chargements sont présentés dans la figure \ref{fig:fig1p14}. On voit ici qu'on obtient les mêmes quatre facteurs qu'avec une rotation varimax orthogonale.

\begin{figure}

{\centering \includegraphics[width=0.7\linewidth]{figures/01-facto-e13} 

}

\caption{Corrélation interfacteurs pour rotation varimax oblique}\label{fig:fig1p13}
\end{figure}

\begin{figure}

{\centering \includegraphics[width=0.7\linewidth]{figures/01-facto-e14} 

}

\caption{Chargements avec rotation oblique varimax}\label{fig:fig1p14}
\end{figure}

\hypertarget{scores-factoriels}{%
\subsection{Scores factoriels}\label{scores-factoriels}}

Avec les données de l'exemple, en nous basant sur les résultats de l'analyse factorielle, nous avons créé quatre nouvelles échelles (une par facteur) que l'on peut calculer pour chaque individu:

\begin{itemize}
\tightlist
\item
  \emph{service} = \((X_4+X_8+X_{11})/3\),
\item
  \emph{produit} = \((X_3+X_6+X_9+X_{12})/4\),
\item
  \emph{paiement} = \((X_2+X_7+X_{10})/3\),
\item
  \emph{prix} = \((X_1+X_5)/2\).
\end{itemize}

Par exemple, la variable \emph{prix} peut donc être vu comme une combinaison linéaire des 12
variables où seulement \(X_1\) et \(X_5\) reçoivent un poids (égal) différent de zéro. Une autre façon de créer de nouvelles variables consiste à calculer des scores factoriels (un pour chaque facteur) pour chaque individu. Par exemple, pour un individu \(i\) donné, le score factoriel pour le facteur \(k\) peut être prédit à l'aide de la formule
\begin{align*}
\hat{F}_{ik} &= \widehat{\mathbf{L}}^{\top}\mathbf{R}^{-1}\boldsymbol{z}\\&=
\widehat{\gamma}_{1, k} z_{i, 1} + \cdots + \widehat{\gamma}_{12, k}z_{i, 12}, 
\end{align*}
où \(z_{i, 1}, \ldots, z_{i, 12}\) sont les valeurs centrées et réduites des observations correspondant à l'individu et où \(\widehat{\gamma}_{1, k}, \ldots, \widehat{\gamma}_{12, k}\) sont des coefficients estimés à partir des chargements \(l_{ij}\) (après rotation) et de la matrice de corrélation des variables \(\mathbf{R}\), avec \(\widehat{\gamma}_{i, k}=\sum_{j=1}^p \hat{l}_{kj}r_{jk}\).

Ainsi, chacune des 12 variables originales contribue au calcul du score
factoriel. Les variables ayant des chargements plus élevés sur ce facteur auront tendance à avoir des poids (\(\widehat{\gamma}\)) plus élevés. Par contre, les scores factoriels ne sont pas uniques car ils dépendent des chargements utilisés (et donc à la fois de la méthode d'estimation et de la méthode de rotation).
On peut également utiliser les scores factoriels au lieu des 12 variables
originales dans des analyses subséquentes. Il est suggéré d'utiliser les nouvelles variables (échelles) obtenues en faisant les moyennes des variables identifiées comme faisant partie de chaque facteur pour les raisons suivantes:

\begin{itemize}
\tightlist
\item
  l'interprétation des scores factoriels est moins claire (chaque facteur dépend de toutes les variables)
\item
  les scores factoriels ne sont pas uniques (ils dépendent de la méthode d'estimation et de rotation).
\item
  les coefficients servant au calcul seront différents d'une étude à l'autre.
\end{itemize}

\begin{figure}

{\centering \includegraphics[width=0.6\linewidth]{figures/01-facto-e15} 

}

\caption{Coefficients du score normalisés}\label{fig:fig1p15}
\end{figure}

Pour obtenir les scores avec \textbf{SAS}, il suffit d'insérer l'option \texttt{score} à la procédure \texttt{factor}. L'option \texttt{out=...} permet de créer un fichier de données \textbf{SAS} qui contient la valeur des \(m\) scores pour chaque individu. Les scores factoriels pour l'exemples sont rapportés dans la figure \ref{fig:fig1p15}. On remarque que:

\begin{itemize}
\tightlist
\item
  pour le premier facteur, trois variables ont des poids importants (\(X_4\), \(X_8\) et \(X_{11}\)). Il s'agit donc d'un facteur très proche du facteur \emph{service}.
\item
  pour le deuxième facteur, les variables \(X_3\), \(X_6\), \(X_9\) et \(X_{12}\) ont des poids importants. Il s'agit donc d'un facteur très proche du facteur \emph{produits}.
\item
  pour le troisième facteur, les variables \(X_2\), \(X_7\), \(X_{10}\) ont des poids importants. Il s'agit donc d'un facteur très proche du facteur \emph{paiement}.
\item
  pour le quatrième facteur, seule la variable \(X_1\) a un poids important.
  On aurait pu s'attendre à ce que ce soit également le cas pour \(X_5\), en lien avec le facteur \emph{prix} --- ce facteur était moins clair selon le alpha de Cronbach.
\end{itemize}

Les corrélations entre les échelles (construites avec les moyennes) et les scores factoriels sont données dans la figure \ref{fig:fig1p16}. On remarque la forte corrélation entre le score factoriel et les échelles construites avec les moyennes pour les facteurs \emph{service}, \emph{produits} et \emph{paiement}. Cela veut dire qu'utiliser les échelles ou les scores factoriels ne devrait pas faire de différence dans des analyses subséquentes. Par contre, cette corrélation est plus faible (0.82) pour le facteur \emph{prix}.

\begin{figure}

{\centering \includegraphics[width=0.65\linewidth]{figures/01-facto-e16} 

}

\caption{Corrélation entre scores et échelles}\label{fig:fig1p16}
\end{figure}

\hypertarget{suxe9lection-de-variables-et-de-moduxe8les-1}{%
\chapter{Sélection de variables et de modèles}\label{suxe9lection-de-variables-et-de-moduxe8les-1}}

\hypertarget{introduction-2}{%
\section{Introduction}\label{introduction-2}}

Ce chapitre présente des principes, outils et méthodes très généraux pour choisir un « bon » modèle. Nous allons principalement utiliser la régression linéaire pour illustrer les méthodes en supposant que tout le monde connaît ce modèle de base. Les méthodes présentées sont en revanche très générales et peuvent être appliquées avec n'importe quel autre modèle (régression logistique, arbres de classification et régression, réseaux de neurones, analyse de survie, etc.)

L'expression « sélection de variables » fait référence à la situation où l'on cherche à sélectionner un sous-ensemble de variables à inclure dans notre modèle à partir d'un ensemble de variables \(X_1, \ldots, X_p\). Le terme variable ici inclut autant des variables distinctes que des transformations d'une ou plusieurs variables.

Par exemple, supposons que les variables \texttt{age}, \texttt{sexe} et \texttt{revenu} sont trois variables explicatives disponibles. Nous pourrions alors considérer choisir entre ces trois variables. Mais aussi, nous pourrions considérer inclure \texttt{age}\({}^2\), \texttt{age}\({}^3\), \(\log(\)\texttt{age}\()\), etc\ldots Nous pourrions aussi considérer des termes d'interactions entre les variables, telles \texttt{age}\(*\)\texttt{revenu} ou \texttt{age}\(*\)\texttt{revenu}\(*\)\texttt{sexe}. Le problème est alors de trouver un bon sous-ensemble de variables parmi toutes celles considérées.

L'expression « sélection de modèle » est un peu plus générale. D'une part, elle inclut la sélection de variables car, pour une famille de modèles spécifiques (régression linéaire par exemple), choisir un sous-ensemble de variables revient à choisir un modèle. D'autre part, elle est plus générale car elle peut aussi faire référence à la situation où l'on cherche à trouver le meilleur modèle parmi des modèles de natures différentes. Par exemple, on pourrait choisir entre une régression linéaire, un arbre de régression, une forêt aléatoire, un réseau de neurones, etc.

\hypertarget{suxe9lection-de-variables-et-de-moduxe8les-selon-les-buts-de-luxe9tude}{%
\section{Sélection de variables et de modèles selon les buts de l'étude}\label{suxe9lection-de-variables-et-de-moduxe8les-selon-les-buts-de-luxe9tude}}

Nous disposons d'une variable réponse \(Y\) et d'un ensemble de variables explicatives \(X_1, \ldots, X_p\). L'attitude à adopter dépend des buts de l'étude.

1e situation : On veut développer un modèle pour faire des prévisions sans qu'il soit important de tester formellement les effets des paramètres individuels.

Dans ce cas, on désire seulement que notre modèle soit performant pour prédire des valeurs futures de \(Y\). On peut alors baser notre choix de variable (et de modèle) en utilisant des outils qui nous guiderons quant aux performances prédictives futures du modèle (voir \(\mathsf{AIC}\), \(\mathsf{BIC}\) et validation-croisée plus loin). On pourra enlever ou rajouter des variables et des transformations de variables au besoin afin d'améliorer les performances prédictives. Les méthodes que nous allons voir concernent eSCEntiellement ce contexte.

2e situation : On veut développer un modèle pour estimer les effets de certaines variables sur notre \(Y\) et tester des hypothèses de recherche spécifiques concernant certaines variables.

Dans ce cas, il est préférable de spécifier le modèle dès le départ selon des considérations scientifiques et de s'en tenir à lui. Faire une sélection de variables dans ce cas est dangereux car on ne peut pas utiliser directement les valeurs-p des tests d'hypothèses (ou les intervalles de confiance sur les paramètres) concernant les paramètres du modèle final car elles ne tiennent pas compte de la variabilité due au processus de sélection de variables.

Une bonne planification de l'étude est alors cruciale afin de collecter les bonnes variables, de spécifier le ou les bons modèles, et de s'assurer d'avoir suffisamment d'observations pour ajuster le ou les modèles désirés.

Si procéder à une sélection de variables est quand même nécessaire dans ce contexte, il est quand même possible de le faire en divisant l'échantillon en deux. La sélection de variables pourrait être alors effectuée avec le premier échantillon. Une fois qu'un modèle est retenu, on pourrait alors réajuster ce modèle avec le deuxième échantillon (sans faire de sélection de variables cette fois-ci). L'inférence sur les paramètres (valeurs-\emph{p}, etc.) sera alors valide. Le désavantage ici qu'il faut avoir une très grande taille d'échantillon au départ afin d'être en mesure de le diviser en deux.

\hypertarget{mieux-vaut-plus-que-moins}{%
\section{Mieux vaut plus que moins}\label{mieux-vaut-plus-que-moins}}

Il est préférable d'avoir un modèle un peu trop complexe qu'un modèle trop simple. Plaçons-nous dans le contexte de la régression linéaire et supposons que le vrai modèle est inclus dans le modèle qui a été ajusté. Il y a donc des variables en trop dans le modèle qui a été ajusté. Le modèle ajusté est surspécifié.

Par exemple, supposons que le vrai modèle est \(Y=\beta_0+\beta_1X_1+\varepsilon\) mais que c'est le modèle \(Y=\beta_0+\beta_1X_1+\beta_2X_2+\varepsilon\)qui a été ajusté. Dans ce cas, règle générale, les estimateurs des paramètres et les prévisions provenant du modèle sont sans biais. Mais leurs variances estimées seront un peu plus élevées car on estime des paramètres pour des variables superflues.

Supposons à l'inverse qu'il manque des variables dans le modèle ajusté et que le modèle ajusté est sous-spécifié. Par exemple, supposons que le vrai modèle est \(Y=\beta_0+\beta_1X_1+\beta_2X_2+\varepsilon\), mais que c'est le modèle \(Y=\beta_0+\beta_1X_1+\varepsilon\) qui a été ajusté. Dans ce cas, généralement, les estimateurs des paramètres et les prévisions sont biaisés.

Ainsi, il est généralement préférable d'avoir un modèle légèrement surspécifié qu'un modèle sous-spécifié. Plus généralement, il est préférable d'avoir un peu trop de variables dans le modèle que de prendre le risque d'omettre une ou plusieurs variables importantes. Encore plus généralement, il est préférable d'avoir un modèle un peu trop complexe que d'avoir un modèle trop simple.

Mais il faut faire attention et ne pas tomber dans l'excès et avoir un modèle trop complexe (avec trop de variables inutiles) car il pourrait souffrir de surajustement (\emph{over-fitting}). Les exemples qui suivent illustreront ce fait.

\hypertarget{trop-beau-pour-uxeatre-vrai}{%
\section{Trop beau pour être vrai}\label{trop-beau-pour-uxeatre-vrai}}

Cette section traite de l'optimisme de l'évaluation d'un modèle lorsqu'on utilise les mêmes données qui ont servies à l'ajuster pour évaluer sa performance. Un principe fondamental lorsque vient le temps d'évaluer la performance prédictive d'un modèle est le suivant : si on utilise les mêmes observations pour évaluer la performance d'un modèle que celles qui ont servi à l'ajuster (à estimer le modèle et ses paramètres), on va surestimer sa performance. Autrement dit, notre estimation de l'erreur que fera le modèle pour prédire des observations futures sera biaisée à la baisse. Ainsi, il aura l'air meilleur que ce qu'il est en réalité. C'est comme si on demandait à un cinéaste d'évaluer son dernier film. Comme c'est son film, il n'aura généralement pas un regard objectif. C'est pourquoi on aura tendance à se fier à l'opinion d'un critique.

On cherchera donc à utiliser des outils et méthodes qui nous donneront l'heure juste (une évaluation objective) quant à la performance prédictive d'un modèle.

\hypertarget{principes-guxe9nuxe9raux}{%
\section{Principes généraux}\label{principes-guxe9nuxe9raux}}

Les idées présentées ici seront illustrées à l'aide de la régression linéaire. Par contre, elles sont valides dans à peu près n'importe quel contexte de modélisation.

Plaçons-nous d'abord dans un contexte plus général que celui de la régression linéaire. Supposons que l'on dispose de \(n\) observations indépendantes sur (\(Y, X_1, \ldots, Xp\)) et que l'on a ajusté un modèle \(\widehat{f}(X_1, \ldots, X_p)\), avec ces données, pour prédire une variable continue \(Y\).

Ce modèle peut être un modèle de régression linéaire,
\begin{align*}
\widehat{f}(X_1, \ldots, X_p) = \widehat{\beta}_0 + \widehat{\beta}_1X_1 + \cdots + \widehat{\beta}_pX_p
\end{align*}
mais il pourrait aussi avoir été construit selon d'autres méthodes (réseau de neurones, arbre de régression, forêt aléatoire, etc.) Une manière de quantifier la performance prédictive du modèle est l'erreur quadratique moyenne de généralisation (\emph{generalization mean squared error}),
\begin{align*}
\mathsf{EMQ}=\mathsf{E}\left[\left\{(Y-\widehat{f}(X_1, \ldots, X_p)\right\}^2\right]
\end{align*}
lorsque (\(Y, X_1, \ldots, X_p\)) est choisi au hasard dans la population. Cette quantité mesure l'erreur (la différence au carré entre la vraie valeur de \(Y\) et la valeur prédite par le modèle) que fait le modèle en moyenne pour l'ensemble de la population. Plus cette quantité est petite, meilleur est le modèle. Le problème est que l'on ne peut pas la calculer, car on ne connaît pas toute la population. Tout au plus peut-on essayer de l'estimer ou bien d'estimer une fonction qui, sans l'estimer directement, classifiera les modèles dans le même ordre qu'elle.

Une première idée est d'estimer \(\mathsf{EMQ}\) avec l'erreur quadratique moyenne de l'échantillon d'apprentissage (\emph{training mean squared error}),
\begin{align*}
\mathsf{EMQ}_a= \frac{1}{n}\sum_{i=1}^n \left\{Y_i-\widehat{f}(X_{i1}, \ldots, X_{ip})\right\}^2.
\end{align*}
Cette quantité est tout simplement l'équivalent du \(\mathsf{EMQ}\), mais est calculée en utilisant seulement notre échantillon.

Malheureusement, selon le principe fondamental de la section précédente, cette quantité n'est pas un bon estimateur de l'\(\mathsf{EMQ}\). En effet, comme on utilise les mêmes observations que celles qui ont estimé le modèle, l'\(\mathsf{EMQ}_a\) aura tendance à toujours diminuer lorsqu'on augmente la complexité du modèle (par exemple, lorsqu'on augmente le nombre de paramètres). Le \(\mathsf{EMQ}_a\) tend à surestimer la qualité du modèle en sous-estimant l'\(\mathsf{EMQ}\). C'est-à-dire, le modèle a l'air meilleur qu'il ne l'est en réalité.

\hypertarget{choix-dun-moduxe8le-polynomial-en-ruxe9gression-linuxe9aire}{%
\subsection{Choix d'un modèle polynomial en régression linéaire}\label{choix-dun-moduxe8le-polynomial-en-ruxe9gression-linuxe9aire}}

Cet exemple simple servira à illustrer le fait qu'on ne peut utiliser directement les mêmes données qui ont servi à ajuster un modèle pour évaluer sa performance.

Nous disposons de 100 observations sur une variable cible \(Y\) et d'une seule variable explicative \(X\). Le fichier \texttt{selection1\_train.xls} contient les données. Nous voulons considérer des modèles polynomiaux (en \(X\)) afin d'en trouver un bon pour prédire \(Y\). Un modèle polynomial est un modèle de la forme \(Y=\beta_0 + \beta_1X+\cdots+\beta_kX^k+\varepsilon\). Le cas \(k=1\) correspond à un modèle linéaire simple, \(k=2\) à un modèle cubique, \(k=3\) à un modèle cubique, etc. Notre but est de déterminer l'ordre (\(k\)) du polynôme qui nous donnera un bon modèle. Voici d'abord le graphe de ces 100 observations.

\begin{Shaded}
\begin{Highlighting}[]
\KeywordTok{setwd}\NormalTok{(}\StringTok{"~/Documents/Dropbox/website/lbelzile.bitbucket.io/MATH60602"}\NormalTok{)}
\NormalTok{train <-}\StringTok{ }\NormalTok{haven}\OperatorTok{::}\KeywordTok{read_sas}\NormalTok{(}\DataTypeTok{data_file =} \StringTok{"selection1_train.sas7bdat"}\NormalTok{)}
\NormalTok{test <-}\StringTok{ }\NormalTok{haven}\OperatorTok{::}\KeywordTok{read_sas}\NormalTok{(}\DataTypeTok{data_file =} \StringTok{"selection1_test.sas7bdat"}\NormalTok{)}
\KeywordTok{library}\NormalTok{(ggplot2)}
\KeywordTok{print}\NormalTok{(}\KeywordTok{ggplot}\NormalTok{(}\DataTypeTok{data =}\NormalTok{ train, }\DataTypeTok{mapping =} \KeywordTok{aes}\NormalTok{(}\DataTypeTok{x=}\NormalTok{x,}\DataTypeTok{y=}\NormalTok{y)) }\OperatorTok{+}\StringTok{ }\KeywordTok{geom_point}\NormalTok{())}
\end{Highlighting}
\end{Shaded}

\includegraphics{MATH60602_files/figure-latex/02-graphedonneestest-1.pdf}

Il s'agit de l'échantillon d'apprentissage. Ces données ont été obtenues par simulation et le vrai modèle sous-jacent (celui qui a généré les données) est le modèle cubique, c'est-à-dire le modèle d'ordre \(k=3\).

Afin de simuler une population, j'ai généré selon le même modèle 100 000 observations supplémentaires. Ces observations ne vont pas servir à estimer les modèles mais seulement à évaluer leur performance afin d'avoir une estimation sans biais. Ces données se trouvent dans \texttt{selection1\_test.xls}

J'ai ajusté tour à tour à tour les modèles polynomiaux jusqu'à l'ordre 10, avec l'échantillon d'apprentissage de taille 100. C'est-à-dire, le modèle linéaire avec un polynôme d'ordre \(k=1\) (linéaire), \(k=2\) (quadratique), etc., jusqu'à \(k=10\). J'ai ensuite obtenu la valeur de l'erreur moyenne quadratique d'apprentissage pour chacun de ces modèles. J'ai ensuite utilisé ces modèles afin de prédire les 100 000 autres observations (la population) et calculé les 100 000 observations de l'échantillon test pour obtenir une très bonne approximation de l'erreur quadratique moyenne de généralisation.

Voici le graphe de l'\(\mathsf{EMQ}_a\) et de l'\(\mathsf{EMQ}\) de généralisation en fonction de l'ordre (\(k\)) du modèle utilisé.

\begin{Shaded}
\begin{Highlighting}[]
\NormalTok{lmkfold <-}\StringTok{ }\ControlFlowTok{function}\NormalTok{(formula, data, k, ...)\{}
\NormalTok{  accu <-}\StringTok{ }\DecValTok{0}
\NormalTok{  k <-}\StringTok{ }\KeywordTok{as.integer}\NormalTok{(k)}
\NormalTok{  n <-}\StringTok{ }\KeywordTok{nrow}\NormalTok{(data)}
\NormalTok{  gp <-}\StringTok{ }\KeywordTok{sample.int}\NormalTok{(n, n, }\DataTypeTok{replace =} \OtherTok{FALSE}\NormalTok{)}
\NormalTok{  folds <-}\StringTok{ }\KeywordTok{split}\NormalTok{(gp, }\KeywordTok{cut}\NormalTok{(}\KeywordTok{seq_along}\NormalTok{(gp), }\DecValTok{10}\NormalTok{, }\DataTypeTok{labels =} \OtherTok{FALSE}\NormalTok{))}
  \ControlFlowTok{for}\NormalTok{(i }\ControlFlowTok{in} \DecValTok{1} \OperatorTok{:}\NormalTok{k)\{}
\NormalTok{   g <-}\StringTok{ }\KeywordTok{as.integer}\NormalTok{(}\KeywordTok{unlist}\NormalTok{(folds[i]))}
\NormalTok{   fitlm <-}\StringTok{ }\KeywordTok{lm}\NormalTok{(formula, data[}\OperatorTok{-}\NormalTok{g,])}
\NormalTok{   accu <-}\StringTok{ }\NormalTok{accu }\OperatorTok{+}\StringTok{ }\KeywordTok{sum}\NormalTok{((data[g, }\KeywordTok{all.vars}\NormalTok{(formula)[}\DecValTok{1}\NormalTok{]] }\OperatorTok{-}\KeywordTok{predict}\NormalTok{(fitlm, }\DataTypeTok{newdata=}\NormalTok{data[g,]))}\OperatorTok{^}\DecValTok{2}\NormalTok{)}
\NormalTok{  \}}
\KeywordTok{return}\NormalTok{(accu}\OperatorTok{/}\NormalTok{n)}
\NormalTok{\}}

\NormalTok{emq <-}\StringTok{ }\KeywordTok{matrix}\NormalTok{(}\DecValTok{0}\NormalTok{, }\DataTypeTok{nrow =} \DecValTok{10}\NormalTok{, }\DataTypeTok{ncol =} \DecValTok{7}\NormalTok{)}
\NormalTok{emqcv <-}\StringTok{ }\KeywordTok{matrix}\NormalTok{(}\DecValTok{0}\NormalTok{, }\DataTypeTok{nrow =} \DecValTok{10}\NormalTok{, }\DataTypeTok{ncol =} \DecValTok{100}\NormalTok{)}
\KeywordTok{library}\NormalTok{(caret)}
\KeywordTok{library}\NormalTok{(ggplot2)}
\ControlFlowTok{for}\NormalTok{(i }\ControlFlowTok{in} \DecValTok{1} \OperatorTok{:}\DecValTok{10}\NormalTok{)\{}
  \KeywordTok{set.seed}\NormalTok{(i}\OperatorTok{*}\DecValTok{1000}\NormalTok{)}
  \CommentTok{# Créer le modèle avec une chaîne de caractère pour le polynôme}
\NormalTok{  meanmod <-}\StringTok{ }\KeywordTok{as.formula}\NormalTok{(}\KeywordTok{paste0}\NormalTok{(}\StringTok{"y~"}\NormalTok{, }\KeywordTok{paste0}\NormalTok{(}\StringTok{"I(x^"}\NormalTok{,}\DecValTok{1} \OperatorTok{:}\NormalTok{i,}\StringTok{")"}\NormalTok{, }\DataTypeTok{collapse=} \StringTok{"+"}\NormalTok{)))}
\NormalTok{  mod <-}\StringTok{  }\KeywordTok{lm}\NormalTok{(meanmod, }\DataTypeTok{data =}\NormalTok{ train)}
  \CommentTok{# Calculer l'erreur moyenne dans les deux échantillons}
\NormalTok{  emq[i,}\DecValTok{1} \OperatorTok{:}\DecValTok{2}\NormalTok{] <-}\StringTok{ }\KeywordTok{c}\NormalTok{(}\KeywordTok{mean}\NormalTok{(}\KeywordTok{resid}\NormalTok{(mod)}\OperatorTok{^}\DecValTok{2}\NormalTok{),}
               \KeywordTok{mean}\NormalTok{((test}\OperatorTok{$}\NormalTok{y }\OperatorTok{-}\KeywordTok{predict}\NormalTok{(mod, }\DataTypeTok{newdata =}\NormalTok{ test))}\OperatorTok{^}\DecValTok{2}\NormalTok{))}
\NormalTok{  emq[i,}\DecValTok{3}\NormalTok{] <-}\StringTok{ }\KeywordTok{summary}\NormalTok{(mod)}\OperatorTok{$}\NormalTok{r.squared}
\NormalTok{  emq[i,}\DecValTok{4}\NormalTok{] <-}\StringTok{ }\KeywordTok{summary}\NormalTok{(mod)}\OperatorTok{$}\NormalTok{adj.r.squared}
\NormalTok{  emq[i,}\DecValTok{5}\NormalTok{] <-}\StringTok{ }\KeywordTok{AIC}\NormalTok{(mod)}
\NormalTok{  emq[i,}\DecValTok{6}\NormalTok{] <-}\StringTok{ }\KeywordTok{BIC}\NormalTok{(mod)}
\CommentTok{# validation croisée avec 10 groupes}
\NormalTok{  emqcv[i,] <-}\StringTok{  }\KeywordTok{replicate}\NormalTok{(}\DataTypeTok{n =}\NormalTok{ 100L, }
                    \KeywordTok{train}\NormalTok{(}\DataTypeTok{form =}\NormalTok{ meanmod, }\DataTypeTok{data =}\NormalTok{ train, }\DataTypeTok{method =} \StringTok{"lm"}\NormalTok{,}
                \DataTypeTok{trControl =} \KeywordTok{trainControl}\NormalTok{(}\DataTypeTok{method =} \StringTok{"cv"}\NormalTok{, }
                                         \DataTypeTok{number =} \DecValTok{10}\NormalTok{))}\OperatorTok{$}\NormalTok{results}\OperatorTok{$}\NormalTok{RMSE}\OperatorTok{^}\DecValTok{2}\NormalTok{)}
\NormalTok{  emq[i,}\DecValTok{7}\NormalTok{] <-}\StringTok{ }\KeywordTok{lmkfold}\NormalTok{(}\DataTypeTok{formula =}\NormalTok{ meanmod, }\DataTypeTok{data =}\NormalTok{ train, }\DataTypeTok{k =} \DecValTok{10}\NormalTok{)}
\NormalTok{\}}



\NormalTok{emqdat <-}\StringTok{ }\KeywordTok{data.frame}\NormalTok{(}\DataTypeTok{ordre =} \KeywordTok{rep}\NormalTok{(}\DecValTok{1} \OperatorTok{:}\DecValTok{10}\NormalTok{, }\DataTypeTok{length.out =} \DecValTok{20}\NormalTok{), }
           \DataTypeTok{emq =} \KeywordTok{c}\NormalTok{(emq[,}\DecValTok{1} \OperatorTok{:}\DecValTok{2}\NormalTok{]),}
           \DataTypeTok{echantillon =} \KeywordTok{factor}\NormalTok{(}\KeywordTok{c}\NormalTok{(}\KeywordTok{rep}\NormalTok{(}\StringTok{"apprentissage"}\NormalTok{,}\DecValTok{10}\NormalTok{), }\KeywordTok{rep}\NormalTok{(}\StringTok{"test"}\NormalTok{, }\DecValTok{10}\NormalTok{)))}
\NormalTok{)}
\KeywordTok{ggplot}\NormalTok{(}\DataTypeTok{data =}\NormalTok{ emqdat, }\KeywordTok{aes}\NormalTok{(}\DataTypeTok{x=}\NormalTok{ordre, }\DataTypeTok{y=}\NormalTok{emq, }\DataTypeTok{color=}\NormalTok{echantillon)) }\OperatorTok{+}
\StringTok{  }\KeywordTok{geom_line}\NormalTok{() }\OperatorTok{+}\StringTok{ }
\StringTok{  }\KeywordTok{geom_point}\NormalTok{(}\KeywordTok{aes}\NormalTok{(}\DataTypeTok{shape=}\NormalTok{echantillon, }\DataTypeTok{color=}\NormalTok{echantillon)) }\OperatorTok{+}\StringTok{ }
\StringTok{  }\KeywordTok{labs}\NormalTok{(}\DataTypeTok{title =} \StringTok{"Erreur moyenne quadratique estimée en fonction de l'ordre"}\NormalTok{, }
     \DataTypeTok{x =} \StringTok{"ordre du polynôme"}\NormalTok{, }
     \DataTypeTok{y =} \StringTok{"erreur moyenne quadratique"}\NormalTok{) }\OperatorTok{+}\StringTok{ }
\StringTok{  }\KeywordTok{theme}\NormalTok{(}\DataTypeTok{legend.position=}\StringTok{"bottom"}\NormalTok{,}
        \DataTypeTok{legend.title=}\KeywordTok{element_blank}\NormalTok{())}
\end{Highlighting}
\end{Shaded}

\begin{center}\includegraphics{MATH60602_files/figure-latex/02-plotEMQa-1} \end{center}

On voit clairement que l'\(\mathsf{EMQ}_a\) diminue en fonction de l'ordre sur l'échantillon d'apprentissage. C'est-à-dire, plus le modèle est complexe, plus l'erreur observée sur l'échantillon d'apprentissage est petite. Mais cela est trompeur. La courbe \(\mathsf{EMQ}\) donne l'heure juste. Il s'agit d'une estimation de la performance réelle des modèles sur de nouvelles données. On voit que le meilleur modèle est donc le modèle cubique (\(k=3\)). Ce qui n'est pas surprenant car il s'agit du modèle que j'ai utilisé pour générer les données. On peut aussi remarquer d'autres éléments intéressants. Premièrement, on obtient un bon gain en performance (\(\mathsf{EMQ}\)) en passant de l'ordre \(2\) à l'ordre \(3\). Ensuite, la perte de performance en passant de l'ordre \(3\) à \(4\), et ensuite à des ordres supérieurs n'est pas si sévère, même si elle est présente. Cela illustre empiriquement qu'il est préférable d'avoir un modèle un peu trop complexe que d'avoir un modèle trop simple. Il serait beaucoup plus grave pour la performance de choisir le modèle avec \(k=2\) que celui avec \(k=4\).

En pratique par contre, on n'a pas accès à la population : les 100 000 observations qui ont servi à estimer l'\(\mathsf{EMQ}\) théorique ne seront pas disponible. Si on a seulement l'échantillon d'apprentissage, soit 100 observations dans notre exemple, comment faire alors pour choisir le bon modèle? C'est ce que nous verrons à partir de la section suivante.

Mais avant cela, nous allons discuter un peu plus en détail au sujet de la régression linéaire et d'une mesure très connue, le coefficient de détermination (\(R^2\)). Supposons que l'on a ajusté un modèle de régression linéaire
\begin{align*}
\widehat{f}(X_1, \ldots, X_p) = \widehat{Y}=\widehat{\beta}_0 + \widehat{\beta}_1X_1+ \cdots + \widehat{\beta}_p X_p.
\end{align*}
La somme du carré des erreurs (\(\mathsf{SCE}\)) pour notre échantillon est
\begin{align*}
\mathsf{SCE}=\sum_{i=1}^n (Y_i - \widehat{\beta}_0 + \widehat{\beta}_1X_1+ \cdots + \widehat{\beta}_p X_p)^2 = \sum_{i=1}^n (Y_i-\widehat{Y}_i)^2.
 \end{align*}
On peut démontrer que si on ajoute une variable quelconque au modèle, la valeur de la somme du carré des erreurs va nécessairement baisser. Il est facile de se convaincre de cela. En régression linéaire, les estimations sont obtenues par la méthode des moindres carrés qui consiste justement à minimiser la \(\mathsf{SCE}\). Ainsi, en ajoutant une variable \(X_{p+1}\) au modèle, la \(\mathsf{SCE}\) ne peut que baisser car, dans le pire des cas, le paramètre de la nouvelle variable sera \(\widehat{\beta}_{p+1}=0\) et on retombera sur le modèle sans cette variable. C'est pourquoi, la quantité \(\mathsf{EMQ}_a=\mathsf{SCE}/n\) ne peut être utilisée comme outil de sélection de modèles en régression linéaire.

Nous venons d'ailleurs d'illustrer cela avec notre exemple sur les modèles polynomiaux. En effet, augmenter l'ordre du polynôme de \(1\) revient à ajouter une variable. Le coefficient de détermination (\(R^2\)) est souvent utilisé comme mesure de qualité du modèle. Il peut s'interpréter comme étant la proportion de la variance de \(Y\) qui est expliquée par le modèle.

Le coefficient de détermination est
\begin{align*}
R^2=\{\mathsf{cor}(\boldsymbol{y}, \widehat{\boldsymbol{y}})\}^2 = 1-\frac{\mathsf{SCE}}{\mathsf{SCT}},
\end{align*}
où \(\mathsf{SCT}=\sum_{i=1}^n (Y_i-\overline{Y})^2\) est la somme des carrés totale calculée en centrant les observations. La somme des carrés totale, \(\mathsf{SCT}\), ne varie pas en fonction du modèle.
Ainsi, on voit que le \(R^2\) va nécessairement augmenter lorsqu'on ajoute une variable au modèle (car la \(\mathsf{SCE}\) diminue). C'est pourquoi on ne peut pas l'utiliser comme outil de sélection de variables.

Le problème principal que nous avons identifié jusqu'à présent afin d'être en mesure de bien estimer la performance d'un modèle est le suivant : si on utilise les mêmes observations pour évaluer la performance d'un modèle que celles qui ont servi à l'ajuster on va surestimer sa performance.

Il existe deux grandes approches pour contourner ce problème lorsque le but est de faire de la sélection de variables ou de modèle :

\begin{itemize}
\tightlist
\item
  utiliser les données de l'échantillon d'apprentissage (en échantillon) et pénaliser \(\mathsf{EMQ}_a\) pour tenir compte de la complexité du modèle (\(\mathsf{AIC}\), \(\mathsf{BIC}\)).
\item
  tenter d'estimer le \(\mathsf{EMQ}\) directement sur d'autres données (hors échantillon) en utilisant des méthodes de rééchantillonnage, notamment la validation croisée et la division de l'échantillon.
\end{itemize}

\hypertarget{crituxe8res-dinformation}{%
\section{Critères d'information}\label{crituxe8res-dinformation}}

Plaçons-nous dans le contexte de la régression linéaire pour l'instant.
Nous avons déjà utilisé les critères \(\mathsf{AIC}\) et \(\mathsf{BIC}\) en analyse factorielle. Il s'agit de mesures qui découlent d'une méthode d'estimation des paramètres, la méthode du maximum de vraisemblance (\emph{maximum likelihood}).

Il s'avère que les estimateurs des paramètres obtenus par la méthode des moindres carrés en régression linéaire sont équivalents à ceux provenant de la méthode du maximum de vraisemblance si on suppose la normalité des termes d'erreurs du modèle. Ainsi, dans ce cas, nous avons accès aux \(\mathsf{AIC}\) et \(\mathsf{BIC}\), deux critères d'information définis pour les modèles dont la fonction objective est la vraisemblance (qui mesure la probabilité des observations sous le modèle postulé suivant une loi choisie par l'utilisateur). La fonction de vraisemblance \(\mathcal{L}\) et la log-vraisemblance \(\ell\) mesurent l'adéquation du modèle.

Supposons que nous avons ajusté un modèle avec \(p\) paramètres en tout (\textbf{incluant} l'ordonnée à l'origine). En régression linéaire, le critère d'information d'Akaike, \(\mathsf{AIC}\), est
\begin{align*}
\mathsf{AIC} &=-2 \ell(\widehat{\boldsymbol{\beta}}, \widehat{\sigma}^2) +2p=n \ln (\mathsf{SCE}) - n\ln(n) + 2p,
\intertext{tandis que le  critère d'information bayésien de Schwartz, $\mathsf{BIC}$, est défini par}
\mathsf{BIC} =-2 \ell(\widehat{\boldsymbol{\beta}}, \widehat{\sigma}^2) + p\ln(n)=n \ln (\mathsf{SCE}) - n\ln(n) + p\ln(n)
\end{align*}
Plus la valeur du \(\mathsf{AIC}\) (ou du \(\mathsf{BIC}\)) est petite, meilleur est l'adéquation. Que se passE-t-il lorsqu'on ajoute un paramètre à un modèle? D'une part, la somme du carré des erreurs va méchaniquement diminuer, et donc la quantité \(n \ln (\mathsf{SCE}/n)\) va diminuer. D'autre part, la valeur de \(p\) augmente de \(1\). Ainsi, le \(\mathsf{AIC}\) peut soit augmenter, soit diminuer, lorsqu'on ajoute un paramètre; idem pour le \(\mathsf{BIC}\). Par exemple, le \(\mathsf{AIC}\) va diminuer seulement si la baisse de la somme du carré des erreurs est suffisante pour compenser le fait que le terme \(2p\) augmente à \(2 (p+1)\).

Ces critères pénalisent l'ajout de variables afin de se prémunir contre le surajustement. De plus, le \(\mathsf{BIC}\) pénalise plus que le \(\mathsf{AIC}\). Par conséquent, le critère \(\mathsf{BIC}\) va choisir des modèles contenant soit le même nombre, soit moins de paramètres que le \(\mathsf{AIC}\).

Les critères \(\mathsf{AIC}\) et \(\mathsf{BIC}\) peuvent être utilisés comme outils de sélection de variables en régression linéaire mais aussi beaucoup plus généralement avec d'autres méthodes basées sur la vraisemblance (analyse factorielle, régression logistique, etc.) En fait, n'importe quel modèle dont les estimateurs proviennent de la méthode du maximum de vraisemblance produira ces quantités. Nous donnerons des formules générales pour le \(\mathsf{AIC}\) et le \(\mathsf{BIC}\) dans le chapitre sur la régression logistique.

Le critère \(\mathsf{BIC}\) est le seul de ces critères qui est convergent. Cela veut dire que si l'ensemble des modèles que l'on considère contient le vrai modèle, alors la probabilité que le critère \(\mathsf{BIC}\) choisissent le bon model tend vers 1 lorsque \(n\) tend vers l'infini. Il faut mettre cela en perspective : il est peu vraisemblable que \(Y\) ait été généré exactement selon un modèle de régression linéaire, car le modèle de régression n'est qu'une approximation de la réalité. Certains auteurs trouvent que le \(\mathsf{BIC}\) est quelquefois trop sévère (il choisit des modèles trop simples) pour les tailles d'échantillons finies. Dans certaines applications, cette parsimonie est utile, mais il n'est pas possible de savoir d'avance lequel de ces deux critères (\(\mathsf{AIC}\) et \(\mathsf{BIC}\)) sera préférable pour un problème donné.

Avant de revenir à l'exemple, voici la description d'une modification du coefficient de détermination, le \(R^2\) ajusté, qui permet (contrairement au \(R^2\)) de faire de la sélection de variables. En régression linéaire, le \(R^2\) ajusté est
\begin{align*}
R^2_a=1-\frac{\mathsf{SCE}/(n-p)}{\mathsf{SCT}/(n-1)}.
\end{align*}
Lorsqu'on ajoute une variable, la somme du carré des erreurs (\(\mathsf{SCE}\)) diminue mais c'est aussi le cas de la quantité \((n-p)\). Ainsi, le \(R^2\) ajusté peut soit augmenter, soit diminuer lorsqu'on ajoute une variable. On peut donc l'utiliser pour choisir le modèle. Plus \(R^2_a\) est élevé, mieux c'est. Ce critère est moins sévère que le \(\mathsf{AIC}\), Ainsi, en général, il va choisir un modèle avec le même nombre ou bien avec plus de paramètres que le \(\mathsf{AIC}\). Pour résumer, on aura la situation suivante :
\[ \#(\mathsf{BIC}) \leq \#(\mathsf{AIC}) \leq \#(R^2_a),\]
où \(\#\) représente le nombre de paramètres du modèle linéaire.

Il est facile d'obtenir les quantités \(R^2_a\), \(\mathsf{AIC}\) et \(\mathsf{BIC}\) avec la procédure \texttt{glmselect} dans \textbf{SAS}. Le fichier \texttt{selection1\_intro.sas} contient les programmes. La sortie qui suit provient des commandes :

\begin{verbatim}
proc glmselect data=multi.selection1_train;
model y=x x*x x*x*x /selection=none;
run;
\end{verbatim}

Il s'agit du modèle cubique (d'ordre 3) en \(x\).

Le tableau qui suit résume ces quantités pour tous les modèles de l'ordre 1 à l'ordre 10.

\begin{Shaded}
\begin{Highlighting}[]
\NormalTok{knitr}\OperatorTok{::}\KeywordTok{kable}\NormalTok{(}\DataTypeTok{x =} \KeywordTok{as.data.frame}\NormalTok{(emq[,}\KeywordTok{c}\NormalTok{(}\DecValTok{2}\NormalTok{,}\DecValTok{1}\NormalTok{,}\DecValTok{3}\OperatorTok{:}\DecValTok{6}\NormalTok{)]), }
             \DataTypeTok{digits =} \DecValTok{2}\NormalTok{, }
             \DataTypeTok{row.names =} \OtherTok{TRUE}\NormalTok{,}
            \DataTypeTok{col.names =} \KeywordTok{c}\NormalTok{(}\StringTok{"}\CharTok{\textbackslash{}\textbackslash{}}\StringTok{(}\CharTok{\textbackslash{}\textbackslash{}}\StringTok{mathsf\{EMQ\}}\CharTok{\textbackslash{}\textbackslash{}}\StringTok{)"}\NormalTok{,}
                          \StringTok{"}\CharTok{\textbackslash{}\textbackslash{}}\StringTok{(}\CharTok{\textbackslash{}\textbackslash{}}\StringTok{mathsf\{EMQ\}_a}\CharTok{\textbackslash{}\textbackslash{}}\StringTok{)"}\NormalTok{,}
                          \StringTok{"}\CharTok{\textbackslash{}\textbackslash{}}\StringTok{(R^2}\CharTok{\textbackslash{}\textbackslash{}}\StringTok{)"}\NormalTok{, }\StringTok{"}\CharTok{\textbackslash{}\textbackslash{}}\StringTok{(R^2_a}\CharTok{\textbackslash{}\textbackslash{}}\StringTok{)"}\NormalTok{,}
                          \StringTok{"}\CharTok{\textbackslash{}\textbackslash{}}\StringTok{(}\CharTok{\textbackslash{}\textbackslash{}}\StringTok{mathsf\{AIC\}}\CharTok{\textbackslash{}\textbackslash{}}\StringTok{)"}\NormalTok{, }\StringTok{"}\CharTok{\textbackslash{}\textbackslash{}}\StringTok{(}\CharTok{\textbackslash{}\textbackslash{}}\StringTok{mathsf\{BIC\}}\CharTok{\textbackslash{}\textbackslash{}}\StringTok{)"}\NormalTok{))}
\end{Highlighting}
\end{Shaded}

\begin{tabular}{l|r|r|r|r|r|r}
\hline
  & \textbackslash{}(\textbackslash{}mathsf\{EMQ\}\textbackslash{}) & \textbackslash{}(\textbackslash{}mathsf\{EMQ\}\_a\textbackslash{}) & \textbackslash{}(R\textasciicircum{}2\textbackslash{}) & \textbackslash{}(R\textasciicircum{}2\_a\textbackslash{}) & \textbackslash{}(\textbackslash{}mathsf\{AIC\}\textbackslash{}) & \textbackslash{}(\textbackslash{}mathsf\{BIC\}\textbackslash{})\\
\hline
1 & 3191.29 & 3674.20 & 0.65 & 0.65 & 1110.70 & 1118.51\\
\hline
2 & 3132.67 & 2879.24 & 0.73 & 0.72 & 1088.32 & 1098.74\\
\hline
3 & 2697.40 & 2620.05 & 0.75 & 0.74 & 1080.88 & 1093.91\\
\hline
4 & 2766.68 & 2581.70 & 0.75 & 0.74 & 1081.41 & 1097.04\\
\hline
5 & 2771.05 & 2580.86 & 0.75 & 0.74 & 1083.38 & 1101.61\\
\hline
6 & 2779.66 & 2577.60 & 0.75 & 0.74 & 1085.25 & 1106.09\\
\hline
7 & 2780.21 & 2577.49 & 0.75 & 0.74 & 1087.24 & 1110.69\\
\hline
8 & 2797.35 & 2531.00 & 0.76 & 0.74 & 1087.42 & 1113.48\\
\hline
9 & 2811.07 & 2527.85 & 0.76 & 0.73 & 1089.30 & 1117.96\\
\hline
10 & 2848.81 & 2519.14 & 0.76 & 0.73 & 1090.95 & 1122.22\\
\hline
\end{tabular}

Les colonnes \(\mathsf{EMQ}\) et \(\mathsf{EMQ}_a\) ont déjà été expliquées à la section précédente et ont été représentées graphiquement.
On voit que le augmente toujours au fur et à mesure qu'on ajoute une variable (augmente l'ordre du polynôme). Les critères \(\mathsf{AIC}\) et \(\mathsf{BIC}\) choisissent le modèle cubique (\(k=3\)), c'est-à-dire le bon modèle. Le \(R^2\) ajusté quant à lui choisit le modèle d'ordre \(4\) (qui est le deuxième meilleur selon le \(\mathsf{EMQ}\)). N'oubliez pas que ces trois critères sont calculés avec l'échantillon d'apprentissage (\(n=100\)), mais en pénalisant l'ajout de variables. On est ainsi en mesure de contrecarrer le problème provenant du fait qu'on ne peut pas utiliser directement le \(\mathsf{EMQ}_a\).

Le \(\mathsf{AIC}\) et le \(\mathsf{BIC}\) sont des critères très utilisés et très généraux. Ils sont disponibles dès qu'on utilise la méthode du maximum de vraisemblance est utilisée comme méthode d'estimation. Le \(R^2\) ajusté a une portée plus limitée car il est spécialisé à la régression linéaire.

\hypertarget{division-de-luxe9chantillon-et-validation-croisuxe9e}{%
\section{Division de l'échantillon et validation croisée}\label{division-de-luxe9chantillon-et-validation-croisuxe9e}}

La deuxième grande approche après celle consistant à pénaliser le \(\mathsf{EMQ}_a\) consiste à tenter d'estimer le \(\mathsf{EMQ}\) directement. Nous allons voir deux telles méthodes ici, la division de l'échantillon et la validation croisée (\emph{cross-validation}).

Ces deux méthodes s'attaquent directement au problème qu'on ne peut utiliser (sans ajustement) les mêmes données qui ont servi à estimer les paramètres d'un modèle pour estimer sa performance. Pour ce faire, l'échantillon de départ est divisé en deux, ou plusieurs parties, qui vont jouer des rôles différents.

\hypertarget{division-de-luxe9chantillon}{%
\subsection{Division de l'échantillon}\label{division-de-luxe9chantillon}}

Cette idée est très simple. Nous avons un échantillon de taille n.~Nous pouvons le diviser au hasard en deux parties de tailles respectives \(n_1\) et \(n_2\) (\(n_1+n_2=n\)),

\begin{itemize}
\tightlist
\item
  un échantillon d'apprentissage (\emph{training}) de taille \(n_1\) et
\item
  un échantillon de validation de taille \(n_2\).
\end{itemize}

L'échantillon d'apprentissage servira à estimer les paramètres du modèle. L'échantillon de validation servira à estimer la performance prédictive (par exemple estimer l'\(\mathsf{EMQ}\)) du modèle. Comme cet échantillon n'a pas servi à estimer le modèle lui-même, il est formé de « nouvelles » observations qui permettent d'évaluer d'une manière réaliste la performance du modèle. Comme il s'agit de nouvelles observations, on n'a pas à pénaliser la complexité du modèle et on peut directement utiliser le critère de performance choisi, par exemple, l'erreur quadratique moyenne, c'est-à-dire, la moyenne des erreurs au carré pour l'échantillon de validation. Cette quantité est une estimation valable de l'\(\mathsf{EMQ}\) de ce modèle. On peut faire la même chose pour tous les modèles en compétition et choisir celui qui a la meilleure performance sur l'échantillon de validation.

Cette approche possède plusieurs avantages. Elle est facile à implanter. Elle est encore plus générale que les critères \(\mathsf{AIC}\) et \(\mathsf{BIC}\). En effet, ces critères découlent de la méthode d'estimation du maximum de vraisemblance. Plusieurs autres types de modèles ne sont pas estimés par la méthode du maximum de vraisemblance (par exemple, les arbres, les forêts aléatoires, les réseaux de neurones, etc.) La performance de ces modèles peut toujours être estimée en divisant l'échantillon. Cette méthode peut donc servir à comparer des modèles de familles différentes. Par exemple, choisit-on un modèle de régression linéaire, une forêt aléatoire ou bien un réseau de neurones?

Cette approche possède tout de même un désavantage. Elle nécessite une grande taille d'échantillon au départ. En effet, comme on divise l'échantillon, on doit en avoir assez pour bien estimer les paramètres du modèle (l'échantillon d'apprentissage) et assez pour bien estimer sa performance (l'échantillon de validation).

La méthode consistant à diviser l'échantillon en deux (apprentissage et validation) afin de sélectionner un modèle est valide. Par contre, si on veut une estimation sans biais de la performance du modèle choisi (celui qui est le meilleur sur l'échantillon de validation), on ne peut pas utiliser directement la valeur observée de l'erreur de ce modèle sur l'échantillon de validation. Elle risque de sous-évaluer l'erreur. En effet, supposons qu'on a 10 échantillons et qu'on ajuste 10 fois le même modèle séparément sur les 10 échantillons. Nous aurons alors 10 estimations différentes de l'erreur du modèle. Il est alors évident que de choisir la plus petite d'entre elles sous-estimerait la vraie erreur du modèle. C'est un peu ce qui se passe lorsqu'on choisit le modèle qui minimise l'erreur sur l'échantillon de validation. Le modèle lui-même est un bon choix, mais l'estimation de son erreur risque d'être sous-évaluée.

Une manière d'avoir une estimation de l'erreur du modèle retenu consiste à diviser l'échantillon de départ en trois (plutôt que deux). Aux échantillons d'apprentissage et de validation, s'ajoute un échantillon « test ». Cet échantillon est laissé de côté durant tout le processus de sélection du modèle qui est effectué avec les deux premiers échantillons tel qu'expliqué plus haut. Une fois un modèle retenu (par exemple celui qui minimise l'erreur sur l'échantillon de validation), on peut alors évaluer sa performance sur l'échantillon test qui n'a pas encore été utilisé jusque là. L'estimation de l'erreur du modèle retenu sera ainsi valide. Il est évident que pour procéder ainsi, on doit avoir une très grande taille d'échantillon au départ.

\hypertarget{validation-croisuxe9e}{%
\subsection{Validation croisée}\label{validation-croisuxe9e}}

Si la taille d'échantillon n'est pas suffisante pour diviser l'échantillon en deux et procéder comme nous venons de l'expliquer, la validation croisée est une bonne alternative. Cette méthode permet d'imiter le processus de division de l'échantillon.

Voici les étapes à suivre pour faire une validation-croisée à \(K\) groupes (\emph{\(K\)-fold cross-validation}) :

\begin{enumerate}
\def\labelenumi{\arabic{enumi}.}
\tightlist
\item
  Diviser l'échantillon au hasard en \(K\) parties \(P_1, P_2, \ldots, P_K\) de taille contenant toutes à peu près le même nombre d'observations.
\item
  Pour \(j = 1\) à \(K\),
\end{enumerate}

\begin{enumerate}
\def\labelenumi{\roman{enumi}.}
\tightlist
\item
  Enlever la partie \(j\).
\item
  Estimer les paramètres du modèle en utilisant les observations des \(K-1\) autres parties combinées.
\item
  Calculer la mesure de performance (par exemple la somme du carré des erreurs) de ce modèle pour le groupe \(P_j\).
\end{enumerate}

\begin{enumerate}
\def\labelenumi{\arabic{enumi}.}
\setcounter{enumi}{2}
\tightlist
\item
  Faire la somme des \(K\) estimations de performance pour obtenir une mesure de performance finale et repondérer au besoin.
\end{enumerate}

On recommande habituellement de prendre entre \(K=5\) et \(10\) groupes (le choix de 10 groupes est celui qui revient le plus souvent en pratique). Si on prend \(K=10\) groupes, alors chaque modèle est estimé avec 90\% des données et on prédit ensuite le 10\% restant. Comme on passe en boucle les 10 parties, chaque observation est prédite une et une seule fois à la fin. Il est important de souligner que les groupes sont formés de façon aléatoire et donc que l'estimé que l'on obtient peut être très variable, surtout si la taille de l'échantillon d'apprentissage est petite. Il arrive également que le modèle ajusté sur un groupe ne puisse pas être utilisé pour prédire les observations mises de côté, notamment si des varibles catégorielles sont présentes. Un échantillonage stratifié permet de pallier à cette lacune, mais ce problème se présente en pratique quand certaines classes ont peu d'observations.

Le cas particulier \(K=n\) (en anglais \emph{leave-one-out cross validation}, ou \(\mathsf{LOOCV}\)) consiste à enlever une seule observation, à estimer le modèle avec les \(n-1\) autres et à valider à l'aide de l'observation laissée de côté et on recommence pour chaque observation. Pour les modèles linéaires, il existe des formules explicites qui nous permettent d'éviter d'ajuster \(n\) régressions par moindre carrés.

Le fichier \texttt{selection3\_cv.sas} contient une macro SAS permettant de faire une validation croisée pour un modèle de régression linéaire.
Revenons à notre exemple où une seule variable explicative est disponible et où l'on cherche à déterminer un bon modèle polynomial.
Voici le même tableau que celui vu plus haut mais avec une colonne en plus, la dernière, \(\mathsf{VC} (K=10)\). Il s'agit des estimations du \(\mathsf{EMQ}\) obtenues avec la validation croisée à 10 groupes. Notez que si vous exécutez le programme, vous n'obtiendrez pas les mêmes valeurs car il y a un élément aléatoire dans ce processus. La colonne représente la moyenne de 100 réplications.

\begin{Shaded}
\begin{Highlighting}[]
\NormalTok{knitr}\OperatorTok{::}\KeywordTok{kable}\NormalTok{(}\DataTypeTok{x =} \KeywordTok{as.data.frame}\NormalTok{(}\KeywordTok{cbind}\NormalTok{(emq[,}\KeywordTok{c}\NormalTok{(}\DecValTok{2}\NormalTok{,}\DecValTok{1}\NormalTok{,}\DecValTok{3}\OperatorTok{:}\DecValTok{6}\NormalTok{)], }\KeywordTok{rowMeans}\NormalTok{(emqcv))), }
             \DataTypeTok{digits =} \DecValTok{2}\NormalTok{, }
             \DataTypeTok{row.names =} \OtherTok{TRUE}\NormalTok{,}
            \DataTypeTok{col.names =} \KeywordTok{c}\NormalTok{(}\StringTok{"}\CharTok{\textbackslash{}\textbackslash{}}\StringTok{(}\CharTok{\textbackslash{}\textbackslash{}}\StringTok{mathsf\{EMQ\}}\CharTok{\textbackslash{}\textbackslash{}}\StringTok{)"}\NormalTok{,}
                          \StringTok{"}\CharTok{\textbackslash{}\textbackslash{}}\StringTok{(}\CharTok{\textbackslash{}\textbackslash{}}\StringTok{mathsf\{EMQ\}_a}\CharTok{\textbackslash{}\textbackslash{}}\StringTok{)"}\NormalTok{,}
                          \StringTok{"}\CharTok{\textbackslash{}\textbackslash{}}\StringTok{(R^2}\CharTok{\textbackslash{}\textbackslash{}}\StringTok{)"}\NormalTok{, }\StringTok{"}\CharTok{\textbackslash{}\textbackslash{}}\StringTok{(R^2_a}\CharTok{\textbackslash{}\textbackslash{}}\StringTok{)"}\NormalTok{,}
                          \StringTok{"}\CharTok{\textbackslash{}\textbackslash{}}\StringTok{(}\CharTok{\textbackslash{}\textbackslash{}}\StringTok{mathsf\{AIC\}}\CharTok{\textbackslash{}\textbackslash{}}\StringTok{)"}\NormalTok{,}
                          \StringTok{"}\CharTok{\textbackslash{}\textbackslash{}}\StringTok{(}\CharTok{\textbackslash{}\textbackslash{}}\StringTok{mathsf\{BIC\}}\CharTok{\textbackslash{}\textbackslash{}}\StringTok{)"}\NormalTok{,}
                          \StringTok{"}\CharTok{\textbackslash{}\textbackslash{}}\StringTok{(}\CharTok{\textbackslash{}\textbackslash{}}\StringTok{mathsf\{VC\} (K=10)}\CharTok{\textbackslash{}\textbackslash{}}\StringTok{)"}\NormalTok{))}
\end{Highlighting}
\end{Shaded}

\begin{tabular}{l|r|r|r|r|r|r|r}
\hline
  & \textbackslash{}(\textbackslash{}mathsf\{EMQ\}\textbackslash{}) & \textbackslash{}(\textbackslash{}mathsf\{EMQ\}\_a\textbackslash{}) & \textbackslash{}(R\textasciicircum{}2\textbackslash{}) & \textbackslash{}(R\textasciicircum{}2\_a\textbackslash{}) & \textbackslash{}(\textbackslash{}mathsf\{AIC\}\textbackslash{}) & \textbackslash{}(\textbackslash{}mathsf\{BIC\}\textbackslash{}) & \textbackslash{}(\textbackslash{}mathsf\{VC\} (K=10)\textbackslash{})\\
\hline
1 & 3191.29 & 3674.20 & 0.65 & 0.65 & 1110.70 & 1118.51 & 3675.37\\
\hline
2 & 3132.67 & 2879.24 & 0.73 & 0.72 & 1088.32 & 1098.74 & 2897.94\\
\hline
3 & 2697.40 & 2620.05 & 0.75 & 0.74 & 1080.88 & 1093.91 & 2675.51\\
\hline
4 & 2766.68 & 2581.70 & 0.75 & 0.74 & 1081.41 & 1097.04 & 2666.16\\
\hline
5 & 2771.05 & 2580.86 & 0.75 & 0.74 & 1083.38 & 1101.61 & 2711.11\\
\hline
6 & 2779.66 & 2577.60 & 0.75 & 0.74 & 1085.25 & 1106.09 & 2757.13\\
\hline
7 & 2780.21 & 2577.49 & 0.75 & 0.74 & 1087.24 & 1110.69 & 2787.95\\
\hline
8 & 2797.35 & 2531.00 & 0.76 & 0.74 & 1087.42 & 1113.48 & 2845.78\\
\hline
9 & 2811.07 & 2527.85 & 0.76 & 0.73 & 1089.30 & 1117.96 & 2895.61\\
\hline
10 & 2848.81 & 2519.14 & 0.76 & 0.73 & 1090.95 & 1122.22 & 2976.04\\
\hline
\end{tabular}

On voit que le modèle cubique (ordre 3) est aussi choisi par la validation croisée, en moyenne (comme il l'était par le \(\mathsf{AIC}\) et le \(\mathsf{BIC}\)). Le graphe qui suit trace les valeurs de l'estimation par validation croisée (courbe de validation croisée) et aussi le \(\mathsf{EMQ}\). On voit que l'estimation par validation croisée suit assez bien la forme du \(\mathsf{EMQ}\) (qu'il est supposé estimer). Les boîtes à moustache permettent d'apprécier la variabilité des estimés de l'erreur moyenne quadratique telles qu'estimée par validation croisée avec 10 groupes.

\begin{Shaded}
\begin{Highlighting}[]
\NormalTok{emqdat <-}\StringTok{ }\KeywordTok{data.frame}\NormalTok{(}\DataTypeTok{ordre =} \KeywordTok{rep}\NormalTok{(}\DecValTok{1}\OperatorTok{:}\DecValTok{10}\NormalTok{, }\DataTypeTok{length.out =} \DecValTok{20}\NormalTok{), }
           \DataTypeTok{emq =} \KeywordTok{c}\NormalTok{(emq[,}\DecValTok{2}\NormalTok{], }\KeywordTok{rowMeans}\NormalTok{(emqcv)),}
           \DataTypeTok{echantillon =} \KeywordTok{factor}\NormalTok{(}\KeywordTok{c}\NormalTok{(}\KeywordTok{rep}\NormalTok{(}\StringTok{"apprentissage"}\NormalTok{,}\DecValTok{10}\NormalTok{), }\KeywordTok{rep}\NormalTok{(}\StringTok{"validation croisée"}\NormalTok{, }\DecValTok{10}\NormalTok{)))}
\NormalTok{)}
\KeywordTok{ggplot}\NormalTok{(}\DataTypeTok{data =}\NormalTok{ emqdat, }\KeywordTok{aes}\NormalTok{(}\DataTypeTok{x=}\NormalTok{ordre, }\DataTypeTok{y=}\NormalTok{emq)) }\OperatorTok{+}
\StringTok{  }\KeywordTok{geom_boxplot}\NormalTok{(}\DataTypeTok{data =} \KeywordTok{data.frame}\NormalTok{(}\DataTypeTok{emq =} \KeywordTok{c}\NormalTok{(}\KeywordTok{t}\NormalTok{(emqcv)), }
                                 \DataTypeTok{ordre =} \KeywordTok{rep}\NormalTok{(}\DecValTok{1}\OperatorTok{:}\DecValTok{10}\NormalTok{, }\DataTypeTok{each=}\DecValTok{100}\NormalTok{)), }\KeywordTok{aes}\NormalTok{(}\DataTypeTok{group=}\NormalTok{ordre),}\DataTypeTok{show.legend =} \OtherTok{FALSE}\NormalTok{) }\OperatorTok{+}
\StringTok{  }\CommentTok{#geom_line(aes(color=color=echantillon)) + }
\StringTok{  }\KeywordTok{geom_point}\NormalTok{(}\KeywordTok{aes}\NormalTok{(}\DataTypeTok{shape=}\NormalTok{echantillon, }\DataTypeTok{color=}\NormalTok{echantillon)) }\OperatorTok{+}\StringTok{ }
\StringTok{  }\KeywordTok{labs}\NormalTok{(}\DataTypeTok{title =} \StringTok{"Erreur moyenne quadratique estimée en fonction de l'ordre"}\NormalTok{, }
     \DataTypeTok{x =} \StringTok{"ordre du polynôme"}\NormalTok{, }
     \DataTypeTok{y =} \StringTok{"erreur moyenne quadratique"}\NormalTok{) }\OperatorTok{+}\StringTok{ }
\StringTok{  }\KeywordTok{theme}\NormalTok{(}\DataTypeTok{legend.position=}\StringTok{"bottom"}\NormalTok{,}
        \DataTypeTok{legend.title=}\KeywordTok{element_blank}\NormalTok{())}
\end{Highlighting}
\end{Shaded}

\begin{center}\includegraphics{MATH60602_files/figure-latex/plotcv-1} \end{center}

\backmatter
  \bibliography{book.bib,packages.bib}

\end{document}
